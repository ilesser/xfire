\chapter{Script Octave de análisis de resultados}
\label{cap:apB}

\begin{lstlisting}[style=C]
% -----------------------------------------------------------------------------
% Name:          rls_analysis.m
% Version:       1.1
% Author:        Federico Damian Camarda
% Date:          09/04/2015
% Description:   Framework to work with the RLS algorithm.
% References:    [1] S. Haykin, Adaptive Filter Theory. 3rd edition pp. 562.
%                    RLS Algorithm.
%                [2] S. Haykin, Adaptive Filter Theory. 3rd edition pp. 598.
%                    QRD RLS Algorithm.
% -----------------------------------------------------------------------------
%
% Block Diagram:
%
%              _________                  _____
%  u(n) ----> |         | ---> d(n) ---> |+    |
%        |    |    W    |                |     |
%        |    |_________|                | sum |
%        |     _________                 |     | ---> e(n)
%        |    |         |                |     |
%        ---> |  W_est  | ---> y(n) ---> |-    |
%             |_________|                |_____|
%
% The u(n) signal is filtered with the W system to generate the desired signal
% d(n). The W system is manually defined.
%
% The RLS algorithm is used to compute the W_est filter, an estimated system
% that, when filtering u(n) with W_est it outputs a signal y(n) that minimizes
% the cost function J(n) = |e(n)|^2 = |d(n) - y(n)|^2
% -----------------------------------------------------------------------------

% Initialize environment.
clc                            % Clean terminal.
clear all                      % Clear environment.
close all                      % Close windows.
diary('rls_analysis.term')     % File where terminal output will be saved on.
diary on                       % Save terminal output on.

% Control parameters.
enable_plots_rls         = 0;
enable_plots_rls_qr      = 0;
enable_plots_rls_qr_hard = 0;
enable_plots_precision   = 1;
enable_debug             = 0;
enable_hard_file         = 1;

% Samples related signals.
N      = 300;                  % Number of samples.
n      = 1:N;                  % Samples' index.

% Filter related signals.
W      = [2 -1 0 0 -1 2]';     % Unknown System coefficients.
N_w    = length(W);            % Filter weigths length.
n_w    = 1:N_w;                % Filter weigths index.
An     = 5000;                 % Multiplying constant.
u      = randn(1, N);          % Input to the filter.
u      = An * u / max(abs(u)); % Normalized input.

% Observed signal.
d      = filter(W, 1, u);

% -----------------------------------------------------------------------------
%                     1. Standard RLS Algorithm : Equations
% -----------------------------------------------------------------------------

% RLS Algorithm implemented by the equations described in [1].
% Zero padding of N_w values is applied to u(n). The reason is that the input
% of the algorithm requries that the first vector is [ 0 0 ... 0 u(n) ],
% the next [ 0 0 ... u(n) u(n-1) ], and so on until the vector is completed.

lambda = 0.9375;                 % RLS forgetting factor.
v      = 1;                      % Initial input variance estimate.
P      = eye([N_w, N_w]) / v;    % Initial inverse correlation matrix.
W_est  = zeros(N_w, 1);          % Initial weights.
u      = [zeros(1, N_w - 1), u]; % System input u(n) with zero padding.
y      = [ ];                    % Output of the estimated system.
e      = [ ];                    % A posteriori error.
xi     = [ ];                    % A priori error. (xi: from the greek letter)

% Open file descriptor to save input vectors.
fid    = fopen('input_vectors_matlab.txt','wt');

% Start algorithm.
for i=1:N                       % All time steps.
    % Take the last N_w samples from the observation.
    U     = u(i + N_w - n_w)';  % e.g. for i=1, => 6,5,4,3,2,1;
                                %      for i=2, => 7,6,5,4,3,2;

    % Save vector in a file.
    fprintf(fid,'%f ',U);
    fprintf(fid,'%f ',d(i));
    fprintf(fid,'\n');

    % Save the estimated output.
    y(i)  = W_est' * U;
    
    % Filter gain.
    K     = (1 / lambda) * P * U / (1 + (1 / lambda) * U' * P * U);

    % Error signal equation. A priori error.
    xi(i) = d(i) - W_est' * U;

    % Filter coefficients.
    W_est = W_est + K * xi(i);

    % Error signal equation. A posteriori error.
    e(i)  = d(i) - W_est' * U;

    % Inverse correlation matrix update.
    P     = (1 / lambda) * P - (1 / lambda) * K * U' * P;
end

% Close file descriptor.
fclose(fid);

% -----------------------------------------------------------------------------
%                      2. Compute the results with hardware
% -----------------------------------------------------------------------------

% Process hardware results if flag is enable.
if(enable_hard_file)
    % Give time to the operator to process the samlpes.
    disp('Samples are ready to be processed.');
    pause

    % Read the hardware results and store them in hard_matrix_16.
    for i=n
       hard_matrix_16(:,:,i)=dlmread('hardware_result_matlab.dat',' ',
                                     [(i-1)*7 0 (i-1)*7+6 6]);
    end

    % Read the software results and store them in soft_matrix_16.
    for i=n
       soft_matrix_16(:,:,i)=dlmread('software_result_matlab.dat',' ',
                                     [(i-1)*7 0 (i-1)*7+6 6]);
    end

    % Mean value of the matrix
    for i=1:N-1
        hard_matrix_mean_value(i) = sum(sum(hard_matrix_16(:,:, i+1)))/49;
        soft_matrix_mean_value(i) = sum(sum(soft_matrix_16(:,:, i+1)))/49;
        error_matrix_mean_value(i) = (sum(sum(soft_matrix_16(:,:, i+1))) - 
                                      sum(sum(hard_matrix_16(:,:, i+1))))/49;
        error_matrix_mean_value_rel(i) = error_matrix_mean_value(i) / 2^(16-1);
        error_matrix_mean_value_per(i) = error_matrix_mean_value_rel(i) * 100;
    end
end

% -----------------------------------------------------------------------------
%                        3. QRD RLS Algorithm : Equations
% -----------------------------------------------------------------------------

% QRD-RLS Algorithm implemented by the matrix equations described in [2].
% Since the QR decomposition gives a lower triangular matrix, and Haykin
% describes an upper triangular matrix, everything is transposed.
%
%  |lambda_sqrt_qr * Phi_sqrt_qr'(n-1)  lambda_sqrt_qr * P_qr(n-1)|  *  Q = 
%  |              U_qr'                               d(n)        |
%
%  |       Phi_sqrt_qr'(n)                         P_qr(n)        |
%  |               0                         xi_qr * gamma_sqrt(n)|
%
lambda_sqrt_qr = sqrt(lambda);    % Square root of the forgetting factor.
Phi_sqrt_qr    = zeros(N_w, N_w); % Square root of the correlation matrix.
P_qr           = zeros(N_w, 1);   % Cross-correlation vector.
W_est_qr       = zeros(N_w, 1);   % Initial weights.
y_qr           = [ ];             % Output of the estimated system.
e_qr           = [ ];             % A posteriori error.
xi_qr          = [ ];             % A priori error. (xi: from the greek letter)
R_max          = [ ];             % Maximum value of the R matrix.

hard_Phi_sqrt_qr    = zeros(N_w, N_w); % Square root of the correlation matrix.
hard_P_qr           = zeros(N_w, 1);   % Cross-correlation vector.
hard_W_est_qr       = zeros(N_w, 1);   % Initial weights.

% Start algorithm.
for i=1:N-1                      % All time steps minus one casue of hard.
    % Take the last N_w samples from the observation.
    U_qr         = u(i + N_w - n_w)';   % for i=1, => 3,2,1; 
                                        % for i=2, => 4,3,2;

    % Save the estimated output.
    y_qr(i)      = W_est_qr' * U_qr;
    
    % Create the input matrix to compute the QR decomposition. This matrix
    % is the result of the concatenation of other matrices as described above:
    A_qr = [ [U_qr' d(i)] ; [lambda_sqrt_qr * Phi_sqrt_qr' lambda_sqrt_qr * P_qr]];

    % Calculate the QR decomposition.
    [Q,R]        = qr(A_qr);

    % Extract the new Phi_sqrt_qr matrix from the R matrix.
    for column = n_w
        Phi_sqrt_qr(column, :) = R(1:N_w, column)';
    end

    % Extract the new P_qr vector from the R matrix.
    P_qr         = R(1:N_w, N_w + 1);

    % Error signal equation. A priori error.
    xi_qr(i)     = d(i) - W_est_qr' * U_qr;

    % Filter coefficients. 
    Phi_sqrt_qr_inv = inv(Phi_sqrt_qr);
    W_est_qr     = (P_qr' * Phi_sqrt_qr_inv)';

    % Error signal equation. A posteriori error.
    e_qr(i)      = d(i) - W_est_qr' * U_qr;

    % Save the maximum value of the R matrix:
    R_max(i)     = max(max(abs(R)));

    if(enable_hard_file)
    % Extract the new Phi_sqrt_qr matrix from the R matrix.
        for column = n_w
            hard_Phi_sqrt_qr(column, :) = hard_matrix_16(1:N_w, column, i+1)';
        end

        % Save the estimated output.
        hard_y_qr(i)      = hard_W_est_qr' * U_qr;
    
        % Extract the new P_qr vector from the R matrix.
        hard_P_qr         = hard_matrix_16(1:N_w, N_w + 1, i+1);
    
        % Error signal equation. A priori error.
        hard_xi_qr(i)     = d(i) - hard_W_est_qr' * U_qr;
    
        % Filter coefficients. 
        hard_Phi_sqrt_qr_inv = inv(hard_Phi_sqrt_qr);
        hard_W_est_qr     = (hard_P_qr' * hard_Phi_sqrt_qr_inv)';
    
        % Error signal equation. A posteriori error.
        hard_e_qr(i)      = d(i) - hard_W_est_qr' * U_qr;
    
        % Save the maximum value of the R matrix:
        hard_R_max(i)     = max(max(abs(hard_matrix_16(:, :, i+1))));

    end

    % Debug.
    if enable_debug
        disp(['Iteration: ', num2str(i)]);
        U_qr
        A_qr
        R
        Phi_sqrt_qr
        P_qr
        W_est_qr
    end
end
\end{lstlisting}
