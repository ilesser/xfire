\chapter{Introducción al trabajo de tesis}

El presente trabajo se encuentra enfocado en el contexto del diseño de hardware digital. El mismo fue motivado por el creciente avance en las nuevas tecnologías implementadas en los sistemas de comunicaciones digitales y en los sistemas \textit{Software Defined Radio}\footnote{\label{SDR}\textit{Software Defined Radio}: Sistema de comunicaciones donde los componentes típicamente implementados en hardware (mezcladores, filtros, amplificadores, moduladores / demoduladores, detectores, etc) son implementados en software}. Los avances en la escala de integración de circuitos integrados ante la constante reducción de las líneas de conducción en los procesos de tratamiento del silicio y el aumento de la frecuencia de operación, hacen que hoy en día sea posible implementar sistemas que en el pasado eran sólo desarrollos teóricos.

En los últimos años, celulares, redes 4G, puntos de acceso de redes inalámbricas tales como Wi-Fi y WiMAX, se encuentran haciendo uso de la tecnología MIMO \footnote{\label{MIMO}\textit{Multiple-input Multiple-output}: Consiste en un sistema de múltiples entradas y salidas, el cual aprovecha fenómenos físicos como la propagación multicamino para incrementar la tasa de transmisión y reducir la tasa de error}. Una de las técnicas para implementar MIMO es el \textit{beamforming} adaptativo, que consiste en la capacidad de separar señales en el dominio del espacio. Este hecho provee un medio para separar la señal deseada de señales de interferencia.

Un \textit{beamformer} adaptativo logra optimizar automáticamente el patrón de un arreglo de múltiples antenas al ajustar el control de los pesos de ponderación hasta que se satisface una determinada función de objetivo preestablecida. Los medios por los cuales esta optimización es lograda están especificados por un algoritmo diseñado para tal propósito. Estos dispositivos utilizan mucha más información disponible en la antena con respecto a un \textit{beamformer} convencional.

El núcleo de un \textit{beamformer} utiliza un sistema de descomposición de matrices para lograr obtener las constantes de ponderación de las señales recibidas en cada antena, siendo la descomposición QR la más utilizada y sobre la cual se basa el presente trabajo.

\section{Objetivo}

La Tesis tiene como objetivo principal diseñar e implementar, en hardware digital, un procesador capaz de resolver un algoritmo utilizado para la descomposición de matrices conocido como RLS-QRD (\textit{Recursive Least Squares - QR Decomposition}). Dicho diseño será realizado a través del uso del lenguaje de descripción de hardware Verilog, y la implementación se obtendrá a través de su síntesis en un dispositivo FPGA \footnote{\label{FPGA}\textit{Field Programmable Gate Array}: dispositivo semiconductor que contiene bloques de lógica cuya interconexión y funcionalidad puede ser configurada ``in situ'' mediante un lenguaje de descripción especializado.}.

El enfoque de la tesis se basará en una parte teórica y una experimental. Se analizarán los modelos y desarrollos algorítmicos para la implementación de una descomposición QR y luego se realizará un análisis para el diseño de un IP core que ponga en práctica uno de ellos. Una vez codificado el hardware, se procederá a realizar su síntesis en un dispositivo FPGA utilizando un kit de desarrollo, se realizarán los bancos de prueba correspondientes y se analizarán los resultados
obtenidos.

El procesador a implementar será sometido a diferentes pruebas con el objetivo de definir parámetros tales como el error, máxima frecuencia de operación, cantidad de operaciones por segundo y consumo de potencia.

\section{Alcance}

Como resultados a obtener de la presente tesis se tienen los siguientes:

\begin{itemize}
    \item IP Core codificado en el lenguaje Verilog de un procesador de descomposición QR.
    \item Resultado de mediciones pertinentes al diseño del procesador.
    \item Análisis comparativo de procesamiento entre el procesador desarrollado y desarrollos de terceros.
    \item Proposición de trabajos futuros y/o mejoras.
\end{itemize}

\newpage

\section{Organización del trabajo}

En esta sección se describe la organización de la presente tesis. Con el objetivo de que la misma sea mínimamente
autocontenida, los primeros capítulos se ocupan de presentar las bases o conocimientos necesarios para comprender la
totalidad del trabajo.

El desarrollo de la tesis se organiza de la siguiente forma:

\begin{itemize}
\item En el capítulo 2 se hará referencia a los conceptos requeridos para comprender la teoría de los sistemas MIMO. Se describirá el contenido teórico requerido para exponer los conceptos de funcionamiento del hardware a desarrollar, el cual se basa en antenas, \textit{beamforming}, criterios para el diseño del filtro adaptativo y algoritmos.
\item En el capítulo 3 se analizarán las diferentes arquitecturas digitales propuestas para implementar el sistema. Se elegirá una de ellas para realizar el desarrollo en base a un determinado criterio.
\item En el capítulo 4 se hará referencia a la implementación de la arquitectura seleccionada en Verilog y se desarrollarán los bancos de prueba de simulación para verificar su correcta funcionalidad. Se generará un IP core en RTL para implementar un sistema MIMO. Dicho RTL cumplirá con ciertas condiciones de portabilidad y legibilidad del código, para que el mismo sea efectivamente un IP core.
\item En el capítulo 5 se experimentará el IP core en un ambiente de simulación y en campo. Se realizará la síntesis del mismo para distintos dispositivos FPGA, y se medirán los recursos utilizados, la máxima frecuencia de operación y la potencia consumida.
\item En el capítulo 6 se extraerán las conclusiones pertinentes sobre los resultados obtenidos y se propondrán futuras mejoras de la arquitecturas a partir del análisis realizado.
\end{itemize}