\subsection{Programable Interrupt Controller}
\label{ssec:pic}
The PIC provides 256 ($2^{8}$) different maskable sources of interruption. Sources are not fixed, and can be
anything that needs to be able to interrupt the microprocessor. Handlers for each interruption are software defined. To be able to handle
an interruption, the PIC must raise its corresponding exception (E30) and the Interruption Exception must be unmasked. The
interruption handler should be able to verify the source of interruption and execute the corresponding subroutine.

\subsubsection{Registers}
\label{sssec:registers}
The PIC provides a set of internal registers used to configure it.

\begin{itemize}
 \item The interrupt cause register (ICR) is 8 bits wide register that holds the address of the interrupting source.
 \item The interrupt mask register \#X (IMRX; X=[0-7]) are eight 32 bits wide registers that provide masking mechanism for every source of
 interruption.
\end{itemize}

The organization of the PIC's registers is shown in table \ref{tbl:pic_reg_organization}

\begin{table}
  \begin{center}
      \begin{bytefield}[bitwidth=0.9em,endianness=big]{40}
      \bitheader{0-7}\\
      \bitbox[]{4}{ICR}      & \bitbox[]{28}{}      & \bitbox{8}{}\\
      \bitbox[]{16}{} \\

      \bitheader{0-31}\\
      \bitbox[]{4}{IMR0}      & \bitbox[]{4}{}      & \bitbox{4}{MS31} & \bitbox{25}{$\hdots$} & \bitbox{3}{MS0}\\
      \bitbox[]{4}{}          & \bitbox[]{4}{}      & \bitbox[rl]{32}{}\\
      \bitbox[]{4}{$\vdots$}  & \bitbox[]{4}{}      & \bitbox[rl]{32}{$\vdots$} \\
      \bitbox[]{4}{}          & \bitbox[]{4}{}      & \bitbox[rl]{32}{}\\
      \bitbox[]{4}{IMR7}      & \bitbox[]{4}{}      & \bitbox{5}{MS255}& \bitbox{22}{$\hdots$} & \bitbox{5}{MS223}\\
      \bitbox[]{40}{} \\
      \end{bytefield}
    \caption{PIC registers organization.}
    \label{tbl:pic_reg_organization}
  \end{center}
\end{table}

\subsubsection{Addressing of Registers}
\label{sssec:pic_reg_addresses}
PIC registers are set and read while talking to the PIC as any other peripheral. PIC registers addressing is shown in table
\ref{tbl:pic_reg_addresses}.

\begin{table}
  \begin{center}
    \begin{tabular}{|c|c|c|}
    \hline
    \textbf{Register Name} & \textbf{Mapped Address} & \textbf{Size}\\
    \hline
    ICR  & 0x00 & 32\\
    IMR0 & 0x01 & 32\\
    IMR1 & 0x02 & 32\\
    IMR2 & 0x03 & 32\\
    IMR3 & 0x04 & 32\\
    IMR4 & 0x05 & 32\\
    IMR5 & 0x06 & 32\\
    IMR6 & 0x07 & 32\\
    IMR7 & 0x08 & 32\\
    \hline
    \end{tabular}
  \caption{PIC registers addressing.}
  \label{tbl:pic_reg_addresses}
  \end{center}
\end{table}

\subsubsection{Ports}
\label{ssec:pic_ports}
\textcolor{red}{Falta definir bien los puertos del PIC para ver como se comunica con los periféricos. Wishbone??}
The PIC ports are shown in table \ref{tbl:pic_ports}.

\begin{table}
  \begin{center}
    \begin{tabular}{|l|l|c|c|}
    \hline
    \textbf{Port name} & \textbf{Description} & \textbf{Size (Bits)} & \textbf{Type}\\
    \hline
    inputs             & Positive edge & 256 & Input \\
    enables            & Enable the peripheral to start talking & 256 & Output \\
    interrupt          & Up when finished and until start is asserted & 1 & Output \\
    \hline
    \end{tabular}
  \end{center}
  \caption{PIC Ports.}
  \label{tbl:pic_ports}
\end{table}

\textcolor{red}{Falta definir el mecanismo de comunicacion de los periféricos}.

\textcolor{red}{Incluir diagrama en bloques del PIC}

