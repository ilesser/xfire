The appication binary interface (ABI) covers details such as:
\begin{itemize}
\item The sizes, layout, and alignment of data types.
\item The calling convention, which controls how functions' arguments are passed and return values retrieved;
for example, whether all parameters are passed on the stack or some are passed in registers, which registers
are used for which function parameters, and whether the first function parameter passed on the stack is pushed
first or last onto the stack.
\item How an application should make system calls to the operating system.
\item The binary format of object files and program libraries.
\end{itemize}

This description is focused on ISO/IEC 9899:1990 C language support.

\section{Scalar data types}

\begin{table}
  \begin{center}
    \begin{tabular}{|c|c|c|c|}
    \hline
    \textbf{C language type} & \textbf{Size of [bytes]} & \textbf{Aligment [bytes]} & \textbf{CPU equivalent}\\
    \hline
    \hline
    char                   & 1  & 1 & byte \\
    \hline
    unsigned char          & 1  & 1 & byte \\
    \hline
    short int              & 2  & 2 & half \\
    \hline
    unsigned short int     & 2  & 2 & half \\
    \hline
    int                    & 4  & 4 & word \\
    \hline
    unsigned int           & 4  & 4 & word \\
    \hline
    long int               & 8  & 4 & double word \\
    \hline
    unsigned long int      & 8  & 4 & double word \\
    \hline
    long long int          & 8  & 4 & double word  \\
    \hline
    unsigned long long int & 8  & 4 & double word  \\
    \hline
    float                  & 4  & 4 & float \\
    \hline
    double                 & 8  & 4 & double \\
    \hline
    long double            & 8  & 4 & double \\
    \hline
    pointer (any type)     & 4  & 4 & word \\
    \hline
    \end{tabular}
  \caption{Data types equivalence.}
  \label{tbl:abi_data_types}
  \end{center}
\end{table}
A null pointer of any type must be zero.


\subsection{Aggregates and Unions}
Aggregates (structures and arrays) and unions assume the alignment of their most strictly aligned element.
\begin{itemize}
\item An array uses the alignment of its elements.
\item Structures and unions, since their elements may be of different types, can require padding to meet alignment restrictions.
      Each element is assigned to the lowest aligned address.
\end{itemize}














\subsection{System calls}
\label{sec:system_calls}
In the following description \emph{ids} is system call ID.


xxxxxxxxx mejorar esto para cuando soporta y no soporta nested exceptions xxxxxxxxxxxxxx

\begin{center}
\begin{tabular}{|r|l|l|l|}
\hline
\textbf{Step} & \textbf{Description} & \textbf{Assembly} & \textbf{Registers Status}\\
\hline
\hline
1. & System call ID (\emph{ids}) is saved in rx & \texttt{        movi rx, ids}    & \hspace{20pt}$rx \leftarrow_{32} ids$ \\
\hline
2. & Jump to the exception handler routine      & \texttt{        trap}            & \hspace{20pt}$l_{esp}[29:0] \leftarrow_{30} pc + 1$ \\
   &                                            &                                  & \hspace{20pt}$pc \leftarrow_{30} 31$ \\
\hline
3. & Jump                                       & \texttt{    31: jump \$trap}     & \hspace{20pt}$pc \leftarrow_{30} trap$ \\
\hline
4. & Jump to system call routine                & \texttt{\$trap: jr rx}           & \hspace{20pt}$pc \leftarrow_{30} pc + ids$ \\
\hline
5. & System call routine                        & System call code                 & \\
\hline
6. & Return from system call                    & \texttt{rfe}                     & $pc \leftarrow_{30} l_{esp}[29:0]$ \\
   &                                            &                                  & $mr[32] \leftarrow_{1} 0$ \\
   &                                            &                                  & $cfg[0] \leftarrow_{1} 1$ \\
\hline
\end{tabular}
\end{center}



\subsection{Subroutine calls}
\label{sec:subroutine_calls}
In the following description \emph{idr} is subroutine ID.

\begin{center}
\begin{tabular}{|r|l|l|l|}
\hline
\textbf{Step} & \textbf{Description} & \textbf{Assembly} & \textbf{Registers Status}\\
\hline
\hline
1. & Subroutine call ID (\emph{idr}) is saved in rx & \texttt{ jal rx, idr}            & \hspace{20pt}$rx \leftarrow_{32} pc+1$ \\
   &                                                &                                  & \hspace{20pt}$pc \leftarrow_{30} pc + idr$ \\
\hline
3. & Subroutine                                     & User code                        & \\
\hline
6. & Return from subroutine                         & \texttt{jr rx}                   & $pc \leftarrow_{30} rx[29:0]$ \\
\hline
\end{tabular}
\end{center}














