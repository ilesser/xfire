\section{Internal Registers}
\label{sec:registers}
The reset state of all the registers is 0 including the special purpose registers. Exception to this rule is the L0
register when implemented, whose reset value is 0x800000000.

\subsection{General Purpose Registers}
\label{ssec:gprs}
\begin{itemize}
   \item There are 32 general purpose registers (GPRs) named R0, R1, $\ldots$ R31.
   \item The GPRs can be used as even-odd pairs holding long integer registers named R0, R2, R4, $\ldots$, R30.
   \item Each GPR has 4 flag bits: carry (C), negative (N), overflow (V), zero (Z) which store the ALU operation status. The negative
         flag is implicit and corresponds to the MSB of the GPR (bit 31).
%    \item When loading data into a GPR from memory, negative (N) and zero (Z) flags are consistent with the data loaded, and carry (C)
%          and overflow (V) are zeroed.
%    \item When an ALU operation writes back double words on pair registers, the even register holding the lower bits of the long data has its carry (C) and
%          overflow (V) flags zeroed; and negative (N) and zero (Z) flags consistent with the data hold by the register, the odd register holding the upper bits of the
%          long data, has all flags consistent with the operation.
   \item When loading data into a GPR from memory, carry (C), overflow(V) and zero (Z) flags are left unmodified; negative (N) flag is implicit in the data
	 loaded in the register.
   \item When an ALU operation writes back double words on pair of registers, the even register holding the lower bits of the long data has its carry (C),
         overflow (V) and zero (Z) flags unmodified; negative (N) flag is implicit in the data hold by the register, the odd register holding the upper bits
         of the long data, has all flags consistent with the operation.
   \item Byte and half data types are loaded into the low portion of GPRs from memory with either zeros or sign bit replicated to fill
         the 32 bits depending on the instruction.
   \item There are 32 single-precision floating point registers (FPRs) named F0, F1, $\ldots$, F31.
   \item The FPRs can be used as even-odd pairs holding double-precision floating point registers named F0, F2, F4, $\ldots$, F30.
   \item Each FPR has 4 flag bits: is-a-number (A), is-infinity (I), negative (N), zero (Z) which store the FPU operation status. The negative
         flag is implicit and corresponds to the MSB of the FPR (bit 31).
%    \item When loading data into a FPR from memory, all flags are consistent with the data loaded.
%    \item When an FPU operation writes back double floats on pair registers, the even register holding the lower bits of the double float has all its flags consistent
%          with the floating point number represented by this register, the odd register holding the upper bits of the double float has all its flags consistent with
%          the operation.
   \item When loading data into a FPR from memory, is-a-number (A), is-infinity (I), and zero (Z) are left unmodified; negative (N) flag is implicit in
         the data loaded in the register.
   \item When an FPU operation writes back double floats on pair registers, the even register holding the lower bits of the double float has its is-a-number
         (A), is-infinity (I), and zero (Z) flags unmodified; negative (N) flag is implicit in the data loaded in the register, the odd register holding the
         upper bits of the double float has all its flags consistent with the operation.
\end{itemize}

\subsection{Special Purpose Registers}
\label{ssec:sprs}
The procesor provides a set of special registers, some of them are accesible with special instructions. Special registers are:
\begin{itemize}
   \item The config register (CFG) is 6 bits wide and stores the configuration status of the processor. Its content is set and read with
         special register instructions. Since special moving instructions transfer 32 bits, bits from 31 down to 6 are discarded on write,
         and return zero on read.

         Bit 0 stores the execution mode. If unset, the processor is running in supervisor mode. If set, it is running in user mode.
         Read section \ref{sec:modes} for more information on execution modes.

         Bit 1 stores the configuration support for nested exceptions. If unset, the processor does not provide support for nested exceptions,
         while if set, it does.
         Read subsection \ref{ssec:exception_handling} for more information on the nested exceptions support.

         Bits 3 down to 2 selects the system tick timer operation mode. Bit 2 selects free running mode. If unset, the system tick timer is in free runing mode. If set
         is in auto stop mode. When running in auto stop mode, bit 3 works as system tick timer start. This bit is not sticky, meaning that it goes back to '0' one
         clock cycle after it was written and always returns '0' on read.
         Read subsection \ref{ssec:sys_tick} for more information on the systick exception.

         Bits 5 down to 4 selects the rounding mode for floating point operations. The encoding is as follows:
         \begin{itemize}
          \item '00’: RN - Round to nearest
          \item '01’: RZ - Round towards zero
          \item '10’: RP - Round towards plus infinity
          \item '11’: RM - Round towards minus infinity
         \end{itemize}
   \item The system timer period register (STP) is a 22 bits register that configures the system timer reload value.
         Read subsection \ref{ssec:sys_tick} for more information on the systick exception.
   \item The mask register (MR) is 33 bits wide and provides the mechanisms to enable or disable the corresponding source of interruption
         or cause. The most significant bit is the global mask bit. This bit is not software accesible; this means that instructions
         cannot read or write to this special bit and its content is set and read automatically by the architecture. When it is set, all
         exceptions are disabled. Its content is set and read with special register instructions. Read section \ref{sec:exceptions}
         for more information on this register.
   \item The instruction register (IR) is 32 bits wide and stores the opcode of the instruction that is being executed. Its content is
         read with special register instructions. A writing operation on this register has no effect.
   \item The system timer register (TMR) is 32 bits wide and corresponds to a 32 bits down counter that generates the systick exception
         when reaches zero. The reload value of the system timer can be configured trough the STP register. This register is read only for software:
         a writing operation has no efect on this register. Read subsection \ref{ssec:sys_tick} for more information on this register.
   \item The peripheral write data register (PWD) is 32 bits wide and provides outgoing communication with peripherals. Its content is
         set and read with special register instructions.
   \item The peripheral write address register (PWA) is 32 bits wide and provides addressing for peripherals. Its content is
         set and read with special register instructions.
   \item The peripheral read data register (PRD) is 32 bits wide and provides incoming communication with peripherals. Its content is
         read with special register instructions. A writing operation on this register has no effect.
   \item The peripheral read address register (PRA) is 32 bits wide and provides incomming identification of peripherals. Its content is read
         with special register instructions. A writing operation on this register has no effect.
   \item The program counter (PC) is 30 bits wide addressing a memory space which ranges from \texttt{0x00000000} to \texttt{0xFFFFFFFC}.
         This register is not software addressable. The current address is given by $4*PC$.
   \item The cause register (CR) is 32 bits wide and stores the cause of exceptions and interrupts. Each bit represents a different
         exception cause / interrupt source. When an exception or interrupt condition is detected, the corresponding bit must be set, and
         shall remain set until the event is handled. The hardware is responsible of reseting the corresponding bit when execution of the
         handler is started. This register is not software addressable. Read section \ref{sec:exceptions} for more information on this register.
   \item The link register stack (LRS) is a set of 32 registers named $L0,L1,\ldots,L31$ which store the return address (PC + 1) value \textbf{only}
         from exceptions. Each register is 36 bits wide. This register file is optional and adds support for nested exceptions. If not supported,
         then one link register (L0) must be implemented. These registers are not software addressable. Read subsection \ref{ssec:exception_handling} for
         more information.
   \item The exception stack pointer (ESP) is a 5 bits wide register and points to the top of the LRS. This register is optional and adds support for
         nested interruptions. Read subsection \ref{ssec:exception_handling} for more information.
\end{itemize}

The structure and organization of registers is shown in Tables \ref{tbl:general_reg_organization} and \ref{tbl:special_reg_organization}.

\begin{figure}
  \begin{center}
    \begin{bytefield}[bitwidth=0.9em,endianness=big]{40}
      \bitheader{0-34}\\
      \bitbox[]{4}{R0}        & \bitbox[r]{1}{}    & \bitbox{1}{C}      & \bitbox{1}{V}      & \bitbox{1}{Z}      & \bitbox{1}{N} & \bitbox[rbt]{31}{}\\
      \bitbox[]{4}{R1}        & \bitbox[r]{1}{}    & \bitbox{1}{C}      & \bitbox{1}{V}      & \bitbox{1}{Z}      & \bitbox{1}{N} & \bitbox[rbt]{31}{}\\
      \bitbox[]{4}{}          & \bitbox[r]{1}{}    & \bitbox[lrt]{1}{}  & \bitbox[lrt]{1}{}  & \bitbox[lrt]{1}{}  & \bitbox[ltr]{1}{} & \bitbox[rt]{31}{} \\
      \bitbox[]{4}{$\vdots$}  & \bitbox[r]{1}{}    & \bitbox[lr]{1}{$\vdots$}        & \bitbox[lr]{1}{$\vdots$} & \bitbox[lr]{1}{$\vdots$} & \bitbox[lr]{1}{$\vdots$} & \bitbox[r]{31}{$\vdots$} \\
      \bitbox[]{4}{}          & \bitbox[r]{1}{}    & \bitbox[lrb]{1}{}  & \bitbox[lrb]{1}{}  & \bitbox[lrb]{1}{}  & \bitbox[lbr]{1}{} & \bitbox[rb]{31}{} \\
      \bitbox[]{4}{R31}       & \bitbox[r]{1}{}    & \bitbox{1}{C}      & \bitbox{1}{V}      & \bitbox{1}{Z}      & \bitbox{1}{N} & \bitbox[rbt]{31}{}\\
      \bitbox[]{40}{} \\

      \bitheader{0-34}\\
      \bitbox[]{4}{F0}        & \bitbox[r]{1}{}      & \bitbox{1}{A}      & \bitbox{1}{I}      & \bitbox{1}{Z}      & \bitbox{1}{N} & \bitbox[rbt]{31}{}\\
      \bitbox[]{4}{F1}        & \bitbox[r]{1}{}      & \bitbox{1}{A}      & \bitbox{1}{I}      & \bitbox{1}{Z}      & \bitbox{1}{N} & \bitbox[rbt]{31}{}\\
      \bitbox[]{4}{}          & \bitbox[r]{1}{}      & \bitbox[lrt]{1}{} & \bitbox[lrt]{1}{}  & \bitbox[lrt]{1}{}  & \bitbox[ltr]{1}{} & \bitbox[rt]{31}{} \\
      \bitbox[]{4}{$\vdots$}  & \bitbox[r]{1}{}      & \bitbox[lr]{1}{$\vdots$}   & \bitbox[lr]{1}{$\vdots$} & \bitbox[lr]{1}{$\vdots$} & \bitbox[lr]{1}{$\vdots$} & \bitbox[r]{31}{$\vdots$} \\
      \bitbox[]{4}{}          & \bitbox[r]{1}{}      & \bitbox[lrb]{1}{} & \bitbox[lrb]{1}{}  & \bitbox[lrb]{1}{}  & \bitbox[lbr]{1}{} & \bitbox[rb]{31}{} \\
      \bitbox[]{4}{F31}       & \bitbox[r]{1}{}      & \bitbox{1}{A}      & \bitbox{1}{I} & \bitbox{1}{Z} & \bitbox{1}{N} & \bitbox[rbt]{31}{}\\
      \bitbox[]{40}{} \\

    \end{bytefield}
  \end{center}
  \caption{General registers structure and organization.}
  \label{fig:general_reg_organization}
\end{figure}

\begin{figure}
  \begin{center}
    \begin{bytefield}[bitwidth=0.9em,endianness=big]{40}

      \bitheader{0-5}\\
      \bitbox[]{4}{CFG}     & \bitbox[]{30}{}     & \bitbox[lrtb]{2}{} & \bitbox[lrtb]{2}{} & \bitbox[lrtb]{1}{}  & \bitbox[lrtb]{1}{}\\
      \bitbox[]{40}{} \\

      \bitheader{0-21}\\
      \bitbox[]{4}{STP}      & \bitbox[]{14}{}      & \bitbox{22}{}\\
      \bitbox[]{40}{} \\

      \bitheader{0-32}\\
      \bitbox[]{4}{MR}     & \bitbox[]{3}{}
                           & \bitbox[lrtb]{1}{} & \bitbox[lrtb]{1}{} & \bitbox[lrtb]{1}{} & \bitbox[lrtb]{1}{} & \bitbox[lrtb]{1}{}
                           & \bitbox[lrtb]{1}{} & \bitbox[lrtb]{1}{} & \bitbox[lrtb]{1}{} & \bitbox[lrtb]{1}{} & \bitbox[lrtb]{1}{}
                           & \bitbox[lrtb]{1}{} & \bitbox[lrtb]{1}{} & \bitbox[lrtb]{1}{} & \bitbox[lrtb]{1}{} & \bitbox[lrtb]{1}{}
                           & \bitbox[lrtb]{1}{} & \bitbox[lrtb]{1}{} & \bitbox[lrtb]{1}{} & \bitbox[lrtb]{1}{} & \bitbox[lrtb]{1}{}
                           & \bitbox[lrtb]{1}{} & \bitbox[lrtb]{1}{} & \bitbox[lrtb]{1}{} & \bitbox[lrtb]{1}{} & \bitbox[lrtb]{1}{}
                           & \bitbox[lrtb]{1}{} & \bitbox[lrtb]{1}{} & \bitbox[lrtb]{1}{} & \bitbox[lrtb]{1}{} & \bitbox[lrtb]{1}{}
                           & \bitbox[lrtb]{1}{} & \bitbox[lrtb]{1}{} & \bitbox[lrtb]{1}{}\\
      \bitbox[]{40}{} \\

      \bitheader{0-31}\\
      \bitbox[]{4}{IR}      & \bitbox[]{4}{}      & \bitbox{32}{}\\
      \bitbox[]{40}{} \\

      \bitheader{0-31}\\
      \bitbox[]{4}{TMR}     & \bitbox[]{4}{}      & \bitbox{32}{}\\
      \bitbox[]{40}{} \\

      \bitheader{0-31}\\
      \bitbox[]{4}{PWD}      & \bitbox[]{4}{}      & \bitbox{32}{}\\
      \bitbox[]{40}{} \\

      \bitheader{0-31}\\
      \bitbox[]{4}{PWA}      & \bitbox[]{4}{}      & \bitbox{32}{}\\
      \bitbox[]{40}{} \\

      \bitheader{0-31}\\
      \bitbox[]{4}{PRD}      & \bitbox[]{4}{}      & \bitbox{32}{}\\
      \bitbox[]{40}{} \\

      \bitheader{0-31}\\
      \bitbox[]{4}{PRA}      & \bitbox[]{4}{}     & \bitbox{32}{}\\
      \bitbox[]{40}{} \\

      \bitheader{0-29}\\
      \bitbox[]{4}{PC}      & \bitbox[]{6}{}      & \bitbox{30}{}\\
      \bitbox[]{40}{} \\

      \bitheader{0-31}\\
      \bitbox[]{4}{CR}     & \bitbox[]{4}{}
                           & \bitbox[lrtb]{1}{} & \bitbox[lrtb]{1}{} & \bitbox[lrtb]{1}{} & \bitbox[lrtb]{1}{} & \bitbox[lrtb]{1}{}
                           & \bitbox[lrtb]{1}{} & \bitbox[lrtb]{1}{} & \bitbox[lrtb]{1}{} & \bitbox[lrtb]{1}{} & \bitbox[lrtb]{1}{}
                           & \bitbox[lrtb]{1}{} & \bitbox[lrtb]{1}{} & \bitbox[lrtb]{1}{} & \bitbox[lrtb]{1}{} & \bitbox[lrtb]{1}{}
                           & \bitbox[lrtb]{1}{} & \bitbox[lrtb]{1}{} & \bitbox[lrtb]{1}{} & \bitbox[lrtb]{1}{} & \bitbox[lrtb]{1}{}
                           & \bitbox[lrtb]{1}{} & \bitbox[lrtb]{1}{} & \bitbox[lrtb]{1}{} & \bitbox[lrtb]{1}{} & \bitbox[lrtb]{1}{}
                           & \bitbox[lrtb]{1}{} & \bitbox[lrtb]{1}{} & \bitbox[lrtb]{1}{} & \bitbox[lrtb]{1}{} & \bitbox[lrtb]{1}{}
                           & \bitbox[lrtb]{1}{} & \bitbox[lrtb]{1}{}\\
      \bitbox[]{40}{} \\

      \bitheader{0-35}\\
      \bitbox[]{4}{L0}       & \bitbox{6}{}             & \bitbox{30}{}\\
      \bitbox[]{4}{L1}       & \bitbox{6}{}             & \bitbox{30}{}\\
      \bitbox[]{4}{}         & \bitbox[lrt]{6}{}        & \bitbox[lrt]{30}{}\\
      \bitbox[]{4}{$\vdots$} & \bitbox[lr]{6}{$\vdots$} & \bitbox[lr]{30}{$\vdots$}\\
      \bitbox[]{4}{}         & \bitbox[lrb]{6}{}        & \bitbox[lrb]{30}{}\\
      \bitbox[]{4}{L31}      & \bitbox{6}{}             & \bitbox{30}{}\\
      \bitbox[]{40}{} \\

      \bitheader{0-4}\\
      \bitbox[]{4}{ESP}     & \bitbox[]{31}{}      & \bitbox{5}{}\\
      \bitbox[]{40}{} \\

    \end{bytefield}
    \caption{Special registers structure and organization.}
    \label{tbl:special_reg_organization}
  \end{center}
\end{figure}


\subsubsection{Addressing of Special Registers}
\label{sssec:sprs_addressing}
Software addressable special registers are set and read with special register instructions called \emph{movi2s} and \emph{movs2i}. For
more informtion on this instructions, read section \ref{sec:isa}.
Special registers addressing is shown in Table \ref{tbl:special_reg_addresses}.

\begin{table}
  \begin{center}
    \begin{tabu} to \textwidth {|X|X[c]|X[c]|X[c]|X|X|}
    \hline
    \rowfont[c]\bfseries
    Register Name & Mapped Address & Hardware Size [bits] & Software Size [bits] & Hardware Access & Software Access \\
    \hline
    \hline
    CFG      & 0x00     & 6        & 32       & Read only      & Read-Write     \\
    \hline
    STP      & 0x01     & 22       & 32       & Read only      & Read-Write     \footnote{10 higher bits return zero on read and are zeroed on write.} \\
    \hline
    MR       & 0x02     & 33       & 32       & Read-Write     & Read-Write     \footnote{Except bit 33 which is non software accessible.}\\
    \hline
    IR       & 0x03     & 32       & 32       & Read-Write     & Read only      \\
    \hline
    TMR      & 0x04     & 32       & 32       & Read-Write     & Read only      \\
    \hline
    PWD      & 0x05     & 32       & 32       & Non accessible & Read-Write     \\
    \hline
    PWA      & 0x06     & 32       & 32       & Non accessible & Read-Write     \\
    \hline
    PRD      & 0x07     & 32       & 32       & Read only      & Read only      \\
    \hline
    PRA      & 0x08     & 32       & 32       & Read only      & Read only      \\
    \hline
    PC       & NA       & 30       & 0        & Read-Write     & Non accessible \\
    \hline
    CR       & NA       & 32       & 0        & Read-Write     & Non accessible \\
    \hline
    L0       & NA       & 36       & 0        & Read-Write     & Non accessible \\
    \hline
    $\vdots$ & $\vdots$ & $\vdots$ & $\vdots$ & $\vdots$       & $\vdots$       \\
    \hline
    L31      & NA       & 36       & 0        & Read-Write     & Non accessible \\
    \hline
    ESP      & NA       & 5        & 0        & Read-Write     & Non accessible \\
    \hline
    \end{tabu}
  \caption{Special registers addressing.}
  \label{tbl:special_reg_addresses}
  \end{center}
\end{table}
