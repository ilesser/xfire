% Cargar CSV con ISA
\DTLsetseparator{|}
\DTLloaddb[]{isa}{latex/isa/isa.csv}
% \DTLloadrawdb[]{isa}{latex/isa/isa.ddb}

\section{Instruction Set Architecture}
\label{sec:isa}

\subsection{Instruction format}
\label{ssec:instruction_format}
There are four intruction main types according to how many registers are used as arguments of the instruction, named Type 0, 1, 2 and 3. The type of
instruction is identified by the two most significant bits of the opcode. This types of instructions are further organized in groups, which identify if
the registers are used as sources or destinations and the role of the remaining bits of the opcode, such as immediate data, displacement, subopcodes or
size of the operation (width of the affected data).

\subsubsection{Type 0 instructions}
\label{sssec:type_0}
Type 0 instructions are encoded with only two bits (bits 28 and 29). Usage of the 28 remaining bits depend on the instruction itself. This type of
instructions are further divided in two groups: A and B. Group A has a 28 bit displacement field. Group B does not specify those bits. The enconding
structure is shown in figure \ref{fig:type_0_encoding_structure}.

\begin{figure}
  \begin{center}
    \begin{bytefield}[endianness=big,bitwidth=0.9em]{40}
       \bitbox[]{8}{Group} & \bitbox[]{32}{Encoding Structure}\\
       \bitbox[]{40}{}\\
       
       \bitbox[]{8}{} & \bitbox[]{2}{2} & \bitbox[]{2}{2} & \bitbox[]{28}{28 bits}\\
       \bitheader{0-31}\\
       \bitbox[]{8}{Group A} & \bitbox{2}{T} & \bitbox{2}{O} & \bitbox{28}{Displacement}\\
       
       \bitbox[]{40}{}\\
       
       \bitbox[]{8}{} & \bitbox[]{2}{2} & \bitbox[]{2}{2} & \bitbox[]{28}{28 bits}\\
       \bitheader{0-31}\\
       \bitbox[]{8}{Group B} & \bitbox{2}{T} & \bitbox{2}{O} & \bitbox{28}{Undefined}\\
    \end{bytefield}
  \end{center}
  \caption{Type 0 instruction encoding structure}
  \label{fig:type_0_encoding_structure}
\end{figure}

\subsubsection{Type 1 instructions}
\label{sssec:type_1}
Type 1 instructions are encoded with four bits (bits 26 to 29) and make use of one register, either as source or destination of the operation. This type
of instructions is further divided in four groups: A, B, C and D. Group A uses a source register (addressed by bits 21 to 25) and a twenty one bits
displacement field (bits 0 to 20). Group B uses a destination register (addressed by bits 16 to 20) and a 16 bits displacement field (bits 0 to 15),
leaving 5 bits unespecified (bits 21 to 25). Group C uses a source register (addressed by bits 21 to 25), a 4 bits subopcode (bits 17 to 20) and a 17 bits
displacement field (bits 0 to 16). Group D uses a destination register (addressed by bits 16 to 20) and a 16 bits immediate field (bits 0 to 15), leaving 5
bits unespecified (bits 21 to 25). The encoding structure is shown in figure \ref{fig:type_1_encoding_structure}.

\begin{figure}
  \begin{center}
    \begin{bytefield}[endianness=big,bitwidth=0.9em]{40}
       \bitbox[]{8}{Group} & \bitbox[]{32}{Encoding Structure}\\
       \bitbox[]{40}{}\\
       
       \bitbox[]{8}{} & \bitbox[]{2}{2} & \bitbox[]{4}{4 bits} & \bitbox[]{5}{5 bits} & \bitbox[]{21}{21 bits}\\
       \bitheader{0-31}\\
       \bitbox[]{8}{Group A} & \bitbox{2}{T} & \bitbox{4}{Opcode} & \bitbox{5}{R src} & \bitbox{21}{Displacement}\\
       
       \bitbox[]{40}{}\\
       
       \bitbox[]{8}{} & \bitbox[]{2}{2} & \bitbox[]{4}{4 bits} & \bitbox[]{5}{5 bits} & \bitbox[]{5}{5 bits} & \bitbox[]{16}{16 bits}\\
       \bitheader{0-31}\\
       \bitbox[]{8}{Group B} & \bitbox{2}{T} & \bitbox{4}{Opcode} & \bitbox{5}{Undefined} & \bitbox{5}{R dst} & \bitbox{16}{Displacement}\\
       
       \bitbox[]{40}{}\\
       
       \bitbox[]{8}{} & \bitbox[]{2}{2} & \bitbox[]{4}{4 bits} & \bitbox[]{5}{5 bits} & \bitbox[]{4}{4 bits} & \bitbox[]{17}{17 bits}\\
       \bitheader{0-31}\\
       \bitbox[]{8}{Group C} & \bitbox{2}{T} & \bitbox{4}{Opcode} & \bitbox{5}{R src} & \bitbox{4}{Sub Op} & \bitbox{17}{Displacement}\\
       
       \bitbox[]{40}{}\\
       
       \bitbox[]{8}{} & \bitbox[]{2}{2} & \bitbox[]{4}{4 bits} & \bitbox[]{5}{5 bits} & \bitbox[]{5}{5 bits} & \bitbox[]{16}{16 bits}\\
       \bitheader{0-31}\\
       \bitbox[]{8}{Group D} & \bitbox{2}{T} & \bitbox{4}{Opcode} & \bitbox{5}{Undefined} & \bitbox{5}{R dst} & \bitbox{16}{Immediate}\\
    \end{bytefield}
  \end{center}
  \caption{Type 1 instruction encoding structure}
  \label{fig:type_1_encoding_structure}
\end{figure}

\subsubsection{Type 2 instructions}
\label{sssec:type_2}
Type 2 instructions are encoded with four bits (bits 26 to 29) and make use of two registers, one being always source and the other one either source or
destination of the operation. This type of instructions are further divided in five groups: A, B, C, D and E. Group A uses a source register (bits 21 to
25), a destination register (bits 16 to 20), a 5 bits subopcode (bits 4 to 8), a 2 bits variation specifier (bits 9 and 10), leaving 9 bits unespecified
(bits 0 to 3 and 11 to 15). Group B uses a source register (bits 21 to 25), a destination register (bits 16 to 20) and a 16 bits displacement field (bits 0
to 15). Group C uses a source register (bits 21 to 25), a destination register (bits 16 to 20) and a 16 bits immediate field (bits 0 to 15). Group D uses
two source registers (bits 21 to 25 and 16 to 20) and a 16 bits displacement field (bits 0 to 15). Group E uses a source register (bits 21 to 25), a
destination register (bits 16 to 20), a two bits subopcode field (bits 14 and 15), a three bits data field (bits 11 to 13), leaving 11 bits unespecified
(bits 0 to 10). The encoding structure is shown in figure \ref{fig:type_2_encoding_structure}.

\begin{figure}
  \begin{center}
    \begin{bytefield}[endianness=big,bitwidth=0.9em]{40}
       \bitbox[]{8}{Group} & \bitbox[]{32}{Encoding Structure}\\
       \bitbox[]{40}{}\\
       
       \bitbox[]{8}{} & \bitbox[]{2}{2} & \bitbox[]{4}{4 bits} & \bitbox[]{5}{5 bits} & \bitbox[]{5}{5 bits} & \bitbox[]{5}{5 bits} & \bitbox[]{2}{2} & \bitbox[]{5}{5 bits} & \bitbox[]{4}{4 bits}\\
       \bitheader{0-31}\\
       \bitbox[]{8}{Group A} & \bitbox{2}{T} & \bitbox{4}{Opcode} & \bitbox{5}{R src} & \bitbox{5}{R dst} & \bitbox{5}{Undefined} & \bitbox{2}{V} & \bitbox{5}{Sub Op} & \bitbox{4}{Undef}\\
       
       \bitbox[]{40}{}\\
       
       \bitbox[]{8}{} & \bitbox[]{2}{2} & \bitbox[]{4}{4 bits} & \bitbox[]{5}{5 bits} & \bitbox[]{5}{5 bits} & \bitbox[]{16}{16 bits}\\
       \bitheader{0-31}\\
       \bitbox[]{8}{Group B} & \bitbox{2}{T} & \bitbox{4}{Opcode} & \bitbox{5}{R src} & \bitbox{5}{R dst} & \bitbox{16}{Displacement}\\
       
       \bitbox[]{40}{}\\
       
       \bitbox[]{8}{} & \bitbox[]{2}{2} & \bitbox[]{4}{4 bits} & \bitbox[]{5}{5 bits} & \bitbox[]{5}{5 bits} & \bitbox[]{16}{16 bits}\\
       \bitheader{0-31}\\
       \bitbox[]{8}{Group C} & \bitbox{2}{T} & \bitbox{4}{Opcode} & \bitbox{5}{R src} & \bitbox{5}{R dst} & \bitbox{16}{Immediate}\\
       
       \bitbox[]{40}{}\\
       
       \bitbox[]{8}{} & \bitbox[]{2}{2} & \bitbox[]{4}{4 bits} & \bitbox[]{5}{5 bits} & \bitbox[]{5}{5 bits} & \bitbox[]{16}{16 bits}\\
       \bitheader{0-31}\\
       \bitbox[]{8}{Group D} & \bitbox{2}{T} & \bitbox{4}{Opcode} & \bitbox{5}{R src 1} & \bitbox{5}{R src 2} & \bitbox{16}{Displacement}\\
       
       \bitbox[]{40}{}\\
       
       \bitbox[]{8}{} & \bitbox[]{2}{2} & \bitbox[]{4}{4 bits} & \bitbox[]{5}{5 bits} & \bitbox[]{5}{5 bits} & \bitbox[]{2}{2} & \bitbox[]{3}{3 bits} & \bitbox[]{11}{11 bits}\\
       \bitheader{0-31}\\
       \bitbox[]{8}{Group E} & \bitbox{2}{T} & \bitbox{4}{Opcode} & \bitbox{5}{R src} & \bitbox{5}{R dst} & \bitbox{2}{SO} & \bitbox{3}{Nib} & \bitbox{11}{Undefined}\\
    \end{bytefield}
  \end{center}
  \caption{Type 2 instruction encoding structure}
  \label{fig:type_2_encoding_structure}
\end{figure}

\subsubsection{Type 3 instructions}
\label{sssec:type_3}
Type 3 instructions make use of three registers, two sources and one destination. This type of instructions are not further divided in groups. All three 
registers are encoded with five bits each (bits 21 to 25 and bits 11 to 15 for source registers and bits 16 to 20 for destination register), a 5 bits
subopcode field (bits 4 to 8), a 2 bits variation specifier (bits 9 and 10), leaving 8 bits undefined (bits 0 to 3 and 26 to 29). The encoding structure is
shown in figure \ref{fig:type_3_encoding_structure}.

\begin{figure}
  \begin{center}
    \begin{bytefield}[endianness=big,bitwidth=0.9em]{40}
       \bitbox[]{8}{Group} & \bitbox[]{32}{Encoding Structure}\\
       \bitbox[]{40}{}\\
       
       \bitbox[]{8}{} & \bitbox[]{2}{2} & \bitbox[]{4}{4 bits} & \bitbox[]{5}{5 bits} & \bitbox[]{5}{5 bits} & \bitbox[]{5}{5 bits} & \bitbox[]{2}{2} & \bitbox[]{5}{5 bits} & \bitbox[]{4}{4 bits}\\
       \bitheader{0-31}\\
       \bitbox[]{8}{All} & \bitbox{2}{T} & \bitbox{4}{Undef} & \bitbox{5}{R src 1} & \bitbox{5}{R dst} & \bitbox{5}{R src 2} & \bitbox{2}{V} & \bitbox{5}{Sub Op} & \bitbox{4}{Undef}\\
    \end{bytefield}
  \end{center}
  \caption{Type 3 instruction encoding structure}
  \label{fig:type_3_encoding_structure}
\end{figure}

\subsection{Full instruction set}
\label{ssec:full_isa}

\subsubsection{Arranged by type}
\label{sssec:isa_by_type}

This sections tabulates instructions organized by type.\\
\begin{itemize}
  \item Type 0 instructions are summarized in table \ref{tbl:type_0}.
  \item Type 1 instructions are summarized in table \ref{tbl:type_1}.
  \item Type 2 instructions are summarized in table \ref{tbl:type_2}.
  \item Type 3 instructions are summarized in table \ref{tbl:type_3}.
\end{itemize}

\begin{table}
  \begin{center}
    \begin{tabu} to \textwidth {llX[l]}
      \hline
      \multicolumn{2}{c}{Mnemonic} & Description \\
      \DTLforeach*[\DTLiseq{\type}{0} \and \DTLiseq{\group}{A}]
        {isa}
	{
	  \mnemonic=mnemonic,
	  \args=args,
	  \shortdescription=shortdescription,
	  \type=type,
	  \group=group}
	{
	  \DTLiffirstrow{\\\hline\hline}{\\} \texttt{\mnemonic} & \texttt{\args} & \shortdescription
	}
      \DTLforeach*[\DTLiseq{\type}{0} \and \DTLiseq{\group}{B}]
        {isa}
	{
	  \mnemonic=mnemonic,
	  \args=args,
	  \shortdescription=shortdescription,
	  \type=type,
	  \group=group}
	{
	  \DTLiffirstrow{\\\hline}{\\} \texttt{\mnemonic} & \texttt{\args} & \shortdescription
	} \\\hline
    \end{tabu}
  \end{center}
\end{table}

\subsubsection{Arranged by function}
\label{sssec:isa_by_function}
This section further tabulates instructions according to their functional behavior, listed as follows:

\begin{itemize}
  \item Register data transfer are summarized in table \ref{tbl:register_data_transfer_instructions}.
  \item Memory data transfer are summarized in table \ref{tbl:memory_data_transfer_instructions}.
  \item Arithmetic are summarized in table \ref{tbl:arithmetic_instructions}.
  \item Cordic are summarized in table \ref{tbl:cordic_instructions}.
  \item Logic are summarized in table \ref{tbl:logic_instructions}.
  \item Control are summarized in table \ref{tbl:control_instructions}.
  \item Data are summarized type conversion in table \ref{tbl:data_type_conversion_instructions}.
\end{itemize}

\begin{table}
  \begin{center}
    \begin{tabu} to \textwidth {|ll|X[l]|}
      \hline
      \multicolumn{2}{|c|}{Mnemonic} & Description
      \DTLforeach*[\DTLiseq{\family}{Register data transfer} \and \DTLiseq{\subfamily}{0}]
	{isa}
	{
	  \mnemonic=mnemonic,
	  \args=args,
	  \description=shortdescription,
	  \family=family,
	  \subfamily=subfamily}
	{
	  \DTLiffirstrow{\\\hline\hline}{\\} \texttt{\mnemonic} & \texttt{\args} & \description
	} 
      \DTLforeach*[\DTLiseq{\family}{Register data transfer} \and \DTLiseq{\subfamily}{1}]
	{isa}
	{
	  \mnemonic=mnemonic,
	  \args=args,
	  \description=shortdescription,
	  \family=family,
	  \subfamily=subfamily}
	{
	  \DTLiffirstrow {\\\hline}{\\} \texttt{\mnemonic} & \texttt{\args} & \description
	}\\\hline
    \end{tabu}
  \caption{Register data transfer instructions}
  \label{tbl:register_data_transfer_instructions}
  \end{center}
\end{table}

\begin{table}
  \begin{center}
    \begin{tabu} to \textwidth {|ll|X[l]|}
      \hline
      \multicolumn{2}{|c|}{Mnemonic} & Description
      \DTLforeach*[\DTLiseq{\family}{Memory data transfer} \and \DTLiseq{\subfamily}{0}]
	{isa}
	{
	  \mnemonic=mnemonic,
	  \args=args,
	  \description=shortdescription,
	  \family=family,
	  \subfamily=subfamily}
	{
	  \DTLiffirstrow{\\\hline\hline}{\\} \texttt{\mnemonic} & \texttt{\args} & \description
	} 
      \DTLforeach*[\DTLiseq{\family}{Memory data transfer} \and \DTLiseq{\subfamily}{1}]
	{isa}
	{
	  \mnemonic=mnemonic,
	  \args=args,
	  \description=shortdescription,
	  \family=family,
	  \subfamily=subfamily}
	{
	  \DTLiffirstrow {\\\hline}{\\} \texttt{\mnemonic} & \texttt{\args} & \description
	}
      \DTLforeach*[\DTLiseq{\family}{Memory data transfer} \and \DTLiseq{\subfamily}{2}]
	{isa}
	{
	  \mnemonic=mnemonic,
	  \args=args,
	  \description=shortdescription,
	  \family=family,
	  \subfamily=subfamily}
	{
	  \DTLiffirstrow {\\\hline}{\\} \texttt{\mnemonic} & \texttt{\args} & \description
	}\\\hline
    \end{tabu}
  \caption{Memory data transfer instructions}
  \label{tbl:memory_data_transfer_instructions}
  \end{center}
\end{table}

\begin{table}
  \begin{center}
    \begin{tabu} to \textwidth {|ll|X[l]|}
      \hline
      \multicolumn{2}{|c|}{Mnemonic} & Description%
      \DTLforeach*[\DTLiseq{\family}{Arithmetic}\and\DTLiseq{\subfamily}{0}]{isa}{
        \mnemonic=mnemonic,
        \args=args,
        \description=shortdescription,
        \family=family,
        \subfamily=subfamily}% Assign list
      {%
        \DTLiffirstrow{\\\hline\hline}{\\}%
        \texttt{\mnemonic} & \texttt{\args} & \description
      }% End loop
      \DTLforeach*[\DTLiseq{\family}{Arithmetic}\and\DTLiseq{\subfamily}{1}]{isa}{%
        \mnemonic=mnemonic,
        \args=args,
        \description=shortdescription,
        \family=family,
        \subfamily=subfamily}% Assign list
      {%
        \DTLiffirstrow{\\\hline}{\\}%
        \texttt{\mnemonic} & \texttt{\args} & \description
      }% End loop
      \DTLforeach*[\DTLiseq{\family}{Arithmetic}\and\DTLiseq{\subfamily}{2}]{isa}{%
        \mnemonic=mnemonic,
        \args=args,
        \description=shortdescription,
        \family=family,
        \subfamily=subfamily}% Assign list
      {%
        \DTLiffirstrow{\\\hline}{\\}%
        \texttt{\mnemonic} & \texttt{\args} & \description
      }% End loop
    \DTLforeach*[\DTLiseq{\family}{Arithmetic}\and\DTLiseq{\subfamily}{3}]{isa}{%
        \mnemonic=mnemonic,
        \args=args,
        \description=shortdescription,
        \family=family,
        \subfamily=subfamily}% Assign list
      {%
        \DTLiffirstrow{\\\hline}{\\}%
        \texttt{\mnemonic} & \texttt{\args} & \description
      }% End loop
      \DTLforeach*[\DTLiseq{\family}{Arithmetic}\and\DTLiseq{\subfamily}{4}]{isa}{%
        \mnemonic=mnemonic,
        \args=args,
        \description=shortdescription,
        \family=family,
        \subfamily=subfamily}% Assign list
     {%
       \DTLiffirstrow{\\\hline}{\\}%
       \texttt{\mnemonic} & \texttt{\args} & \description
     }% End loop
     \DTLforeach*[\DTLiseq{\family}{Arithmetic}\and\DTLiseq{\subfamily}{5}]{isa}{%
       \mnemonic=mnemonic,
       \args=args,
       \description=shortdescription,
       \family=family,
       \subfamily=subfamily}% Assign list
     {%
       \DTLiffirstrow{\\\hline}{\\}%
       \texttt{\mnemonic} & \texttt{\args} & \description
     }% End loop 
     \DTLforeach*[\DTLiseq{\family}{Arithmetic}\and\DTLiseq{\subfamily}{6}]{isa}{%
       \mnemonic=mnemonic,
       \args=args,
       \description=shortdescription,
       \family=family,
       \subfamily=subfamily}% Assign list
     {%
       \DTLiffirstrow{\\\hline}{\\}%
       \texttt{\mnemonic} & \texttt{\args} & \description
     }% End loop
     \DTLforeach*[\DTLiseq{\family}{Arithmetic}\and\DTLiseq{\subfamily}{7}]{isa}{%
       \mnemonic=mnemonic,
       \args=args,
       \description=shortdescription,
       \family=family,
       \subfamily=subfamily}% Assign list
     {%
       \DTLiffirstrow{\\\hline}{\\}%
       \texttt{\mnemonic} & \texttt{\args} & \description
     }% End loop
     \DTLforeach*[\DTLiseq{\family}{Arithmetic}\and\DTLiseq{\subfamily}{8}]{isa}{%
       \mnemonic=mnemonic,
       \args=args,
       \description=shortdescription,
       \family=family,
       \subfamily=subfamily}% Assign list
     {%
       \DTLiffirstrow{\\\hline}{\\}%
       \texttt{\mnemonic} & \texttt{\args} & \description
     }% End loop
     \DTLforeach*[\DTLiseq{\family}{Arithmetic}\and\DTLiseq{\subfamily}{9}]{isa}{%
       \mnemonic=mnemonic,
       \args=args,
       \description=shortdescription,
       \family=family,
       \subfamily=subfamily} % Assign list
     {%
       \DTLiffirstrow{\\\hline}{\\}%
       \texttt{\mnemonic} & \texttt{\args} & \description
     }% End loop
     \DTLforeach*[\DTLiseq{\family}{Arithmetic}\and\DTLiseq{\subfamily}{10}]{isa}{%
        \mnemonic=mnemonic,
        \args=args,
        \description=shortdescription,
        \family=family,
        \subfamily=subfamily}% Assign list
      {%
        \DTLiffirstrow{\\\hline}{\\}%
        \texttt{\mnemonic} & \texttt{\args} & \description
      }% End loop
      \\\hline
    \end{tabu}
  \caption{Arithmetic instructions}
  \label{tbl:arithmetic_instructions}
  \end{center}
\end{table}

\begin{table}
  \begin{center}
    \begin{tabu} to \textwidth {|ll|X[l]|}
      \hline
      \multicolumn{2}{|c|}{Mnemonic} & Description
      \DTLforeach*[\DTLiseq{\family}{Arithmetic} \and \DTLiseq{\subfamily}{0}]{isa}{
	  \mnemonic=mnemonic,
	  \args=args,
	  \description=shortdescription,
	  \family=family,
	  \subfamily=subfamily}
	{
	  \DTLiffirstrow{\\\hline\hline}{\\}
	  \texttt{\mnemonic} & \texttt{\args} & \description
	}
      \DTLforeach*[\DTLiseq{\family}{Arithmetic} \and \DTLiseq{\subfamily}{1}]{isa}{
	  \mnemonic=mnemonic,
	  \args=args,
	  \description=shortdescription,
	  \family=family,
	  \subfamily=subfamily}
	{
	  \DTLiffirstrow{\\\hline}{\\}
	  \texttt{\mnemonic} & \texttt{\args} & \description
	}
      \DTLforeach*[\DTLiseq{\family}{Arithmetic} \and \DTLiseq{\subfamily}{2}]{isa}{
	  \mnemonic=mnemonic,
	  \args=args,
	  \description=shortdescription,
	  \family=family,
	  \subfamily=subfamily}
	{
	  \DTLiffirstrow{\\\hline}{\\}
	  \texttt{\mnemonic} & \texttt{\args} & \description
	}
      \DTLforeach*[\DTLiseq{\family}{Arithmetic} \and \DTLiseq{\subfamily}{3}]{isa}{
	  \mnemonic=mnemonic,
	  \args=args,
	  \description=shortdescription,
	  \family=family,
	  \subfamily=subfamily}
	{
	  \DTLiffirstrow{\\\hline}{\\}
	  \texttt{\mnemonic} & \texttt{\args} & \description
	}
      \DTLforeach*[\DTLiseq{\family}{Arithmetic} \and \DTLiseq{\subfamily}{4}]{isa}{
	  \mnemonic=mnemonic,
	  \args=args,
	  \description=shortdescription,
	  \family=family,
	  \subfamily=subfamily}
	{
	  \DTLiffirstrow{\\\hline}{\\}
	  \texttt{\mnemonic} & \texttt{\args} & \description
	}
      \DTLforeach*[\DTLiseq{\family}{Arithmetic} \and \DTLiseq{\subfamily}{5}]{isa}{
	  \mnemonic=mnemonic,
	  \args=args,
	  \description=shortdescription,
	  \family=family,
	  \subfamily=subfamily}
	{
	  \DTLiffirstrow{\\\hline}{\\}
	  \texttt{\mnemonic} & \texttt{\args} & \description
	}
      \DTLforeach*[\DTLiseq{\family}{Arithmetic} \and \DTLiseq{\subfamily}{6}]{isa}{
	  \mnemonic=mnemonic,
	  \args=args,
	  \description=shortdescription,
	  \family=family,
	  \subfamily=subfamily}
	{
	  \DTLiffirstrow{\\\hline}{\\}
	  \texttt{\mnemonic} & \texttt{\args} & \description
	}
      \DTLforeach*[\DTLiseq{\family}{Arithmetic} \and \DTLiseq{\subfamily}{7}]{isa}{
	  \mnemonic=mnemonic,
	  \args=args,
	  \description=shortdescription,
	  \family=family,
	  \subfamily=subfamily}
	{
	  \DTLiffirstrow{\\\hline}{\\}
	  \texttt{\mnemonic} & \texttt{\args} & \description
	}
      \DTLforeach*[\DTLiseq{\family}{Arithmetic} \and \DTLiseq{\subfamily}{8}]{isa}{
	  \mnemonic=mnemonic,
	  \args=args,
	  \description=shortdescription,
	  \family=family,
	  \subfamily=subfamily}
	{
	  \DTLiffirstrow{\\\hline}{\\}
	  \texttt{\mnemonic} & \texttt{\args} & \description
	}
      \DTLforeach*[\DTLiseq{\family}{Arithmetic} \and \DTLiseq{\subfamily}{9}]{isa}{
	  \mnemonic=mnemonic,
	  \args=args,
	  \description=shortdescription,
	  \family=family,
	  \subfamily=subfamily}
	{
	  \DTLiffirstrow{\\\hline}{\\}
	  \texttt{\mnemonic} & \texttt{\args} & \description
	}
      \DTLforeach*[\DTLiseq{\family}{Arithmetic} \and \DTLiseq{\subfamily}{10}]{isa}{
	  \mnemonic=mnemonic,
	  \args=args,
	  \description=shortdescription,
	  \family=family,
	  \subfamily=subfamily}
	{
	  \DTLiffirstrow{\\\hline}{\\}
	  \texttt{\mnemonic} & \texttt{\args} & \description
	}\\\hline
    \end{tabu}
  \caption{Arithmetic instructions}
  \label{tbl:arithmetic_instructions}
  \end{center}
\end{table}

\begin{center}
  \begin{longtabu} to \textwidth {|ll|X[l]|}
  \caption{Cordic instructions}
  \label{tbl:cordic_instructions}\\
    \hline
    \multicolumn{2}{|c|}{Mnemonic} & Description
    \\\hline
    \endfirsthead
    \hline
    \multicolumn{3}{|c|}{-- Continued from previous page}
    \\\hline
    \multicolumn{2}{|c|}{Mnemonic} & Description
    \\\hline\hline
    \endhead
    \hline \multicolumn{3}{|r|}{{Continued on next page}}
    \\\hline
    \endfoot
    \\\hline
    \endlastfoot
    \DTLforeach*[\DTLiseq{\family}{Cordic} \and \DTLiseq{\subfamily}{0}]{isa}{
	\mnemonic=mnemonic,
	\args=args,
	\description=shortdescription,
	\family=family,
	\subfamily=subfamily}
      {
	\DTLiffirstrow{}{\\}
	\texttt{\mnemonic} & \texttt{\args} & \description
      }
    \DTLforeach*[\DTLiseq{\family}{Cordic} \and \DTLiseq{\subfamily}{1}]{isa}{
	\mnemonic=mnemonic,
	\args=args,
	\description=shortdescription,
	\family=family,
	\subfamily=subfamily}
      {
	\DTLiffirstrow{\\\hline}{\\}
	\texttt{\mnemonic} & \texttt{\args} & \description
      }
    \DTLforeach*[\DTLiseq{\family}{Cordic} \and \DTLiseq{\subfamily}{2}]{isa}{
	\mnemonic=mnemonic,
	\args=args,
	\description=shortdescription,
	\family=family,
	\subfamily=subfamily}
      {
	\DTLiffirstrow{\\\hline}{\\}
	\texttt{\mnemonic} & \texttt{\args} & \description
      }
    \DTLforeach*[\DTLiseq{\family}{Cordic} \and \DTLiseq{\subfamily}{3}]{isa}{
	\mnemonic=mnemonic,
	\args=args,
	\description=shortdescription,
	\family=family,
	\subfamily=subfamily}
      {
	\DTLiffirstrow{\\\hline}{\\}
	\texttt{\mnemonic} & \texttt{\args} & \description
      }
    \DTLforeach*[\DTLiseq{\family}{Cordic} \and \DTLiseq{\subfamily}{4}]{isa}{
	\mnemonic=mnemonic,
	\args=args,
	\description=shortdescription,
	\family=family,
	\subfamily=subfamily}
      {
	\DTLiffirstrow{\\\hline}{\\}
	\texttt{\mnemonic} & \texttt{\args} & \description
      }
    \DTLforeach*[\DTLiseq{\family}{Cordic} \and \DTLiseq{\subfamily}{5}]{isa}{
	\mnemonic=mnemonic,
	\args=args,
	\description=shortdescription,
	\family=family,
	\subfamily=subfamily}
      {
	\DTLiffirstrow{\\\hline}{\\}
	\texttt{\mnemonic} & \texttt{\args} & \description
      }
    \DTLforeach*[\DTLiseq{\family}{Cordic} \and \DTLiseq{\subfamily}{6}]{isa}{
	\mnemonic=mnemonic,
	\args=args,
	\description=shortdescription,
	\family=family,
	\subfamily=subfamily}
      {
	\DTLiffirstrow{\\\hline}{\\}
	\texttt{\mnemonic} & \texttt{\args} & \description
      }
    \DTLforeach*[\DTLiseq{\family}{Cordic} \and \DTLiseq{\subfamily}{7}]{isa}{
	\mnemonic=mnemonic,
	\args=args,
	\description=shortdescription,
	\family=family,
	\subfamily=subfamily}
      {
	\DTLiffirstrow{\\\hline}{\\}
	\texttt{\mnemonic} & \texttt{\args} & \description
      }
    \DTLforeach*[\DTLiseq{\family}{Cordic} \and \DTLiseq{\subfamily}{8}]{isa}{
	\mnemonic=mnemonic,
	\args=args,
	\description=shortdescription,
	\family=family,
	\subfamily=subfamily}
      {
	\DTLiffirstrow{\\\hline}{\\}
	\texttt{\mnemonic} & \texttt{\args} & \description
      }
    \DTLforeach*[\DTLiseq{\family}{Cordic} \and \DTLiseq{\subfamily}{9}]{isa}{
	\mnemonic=mnemonic,
	\args=args,
	\description=shortdescription,
	\family=family,
	\subfamily=subfamily}
      {
	\DTLiffirstrow{\\\hline}{\\}
	\texttt{\mnemonic} & \texttt{\args} & \description
      }
    \DTLforeach*[\DTLiseq{\family}{Cordic} \and \DTLiseq{\subfamily}{10}]{isa}{
	\mnemonic=mnemonic,
	\args=args,
	\description=shortdescription,
	\family=family,
	\subfamily=subfamily}
      {
	\DTLiffirstrow{\\\hline}{\\}
	\texttt{\mnemonic} & \texttt{\args} & \description
      }
    \DTLforeach*[\DTLiseq{\family}{Cordic} \and \DTLiseq{\subfamily}{11}]{isa}{
	\mnemonic=mnemonic,
	\args=args,
	\description=shortdescription,
	\family=family,
	\subfamily=subfamily}
      {
	\DTLiffirstrow{\\\hline}{\\}
	\texttt{\mnemonic} & \texttt{\args} & \description
      }
    \DTLforeach*[\DTLiseq{\family}{Cordic} \and \DTLiseq{\subfamily}{12}]{isa}{
	\mnemonic=mnemonic,
	\args=args,
	\description=shortdescription,
	\family=family,
	\subfamily=subfamily}
      {
	\DTLiffirstrow{\\\hline}{\\}
	\texttt{\mnemonic} & \texttt{\args} & \description
      }
    \DTLforeach*[\DTLiseq{\family}{Cordic} \and \DTLiseq{\subfamily}{13}]{isa}{
	\mnemonic=mnemonic,
	\args=args,
	\description=shortdescription,
	\family=family,
	\subfamily=subfamily}
      {
	\DTLiffirstrow{\\\hline}{\\}
	\texttt{\mnemonic} & \texttt{\args} & \description
      }
    \DTLforeach*[\DTLiseq{\family}{Cordic} \and \DTLiseq{\subfamily}{14}]{isa}{
	\mnemonic=mnemonic,
	\args=args,
	\description=shortdescription,
	\family=family,
	\subfamily=subfamily}
      {
	\DTLiffirstrow{\\\hline}{\\}
	\texttt{\mnemonic} & \texttt{\args} & \description
      }
    \DTLforeach*[\DTLiseq{\family}{Cordic} \and \DTLiseq{\subfamily}{15}]{isa}{
	\mnemonic=mnemonic,
	\args=args,
	\description=shortdescription,
	\family=family,
	\subfamily=subfamily}
      {
	\DTLiffirstrow{\\\hline}{\\}
	\texttt{\mnemonic} & \texttt{\args} & \description
      }
  \end{longtabu}
\end{center}

\begin{table}
  \begin{center}
    \begin{tabu} to \textwidth {|ll|X[l]|}
      \hline
      \multicolumn{2}{|c|}{Mnemonic} & Description
      \DTLforeach*[\DTLiseq{\family}{Logic} \and \DTLiseq{\subfamily}{0}]
	{isa}
	{
	  \mnemonic=mnemonic,
	  \args=args,
	  \description=shortdescription,
	  \family=family,
	  \subfamily=subfamily}
	{
	  \DTLiffirstrow{\\\hline\hline}{\\} \texttt{\mnemonic} & \texttt{\args} & \description
	} 
      \DTLforeach*[\DTLiseq{\family}{Logic} \and \DTLiseq{\subfamily}{1}]
	{isa}
	{
	  \mnemonic=mnemonic,
	  \args=args,
	  \description=shortdescription,
	  \family=family,
	  \subfamily=subfamily}
	{
	  \DTLiffirstrow {\\\hline}{\\} \texttt{\mnemonic} & \texttt{\args} & \description
	}
      \DTLforeach*[\DTLiseq{\family}{Logic} \and \DTLiseq{\subfamily}{2}]
	{isa}
	{
	  \mnemonic=mnemonic,
	  \args=args,
	  \description=shortdescription,
	  \family=family,
	  \subfamily=subfamily}
	{
	  \DTLiffirstrow {\\\hline}{\\} \texttt{\mnemonic} & \texttt{\args} & \description
	}
      \DTLforeach*[\DTLiseq{\family}{Logic} \and \DTLiseq{\subfamily}{3}]
	{isa}
	{
	  \mnemonic=mnemonic,
	  \args=args,
	  \description=shortdescription,
	  \family=family,
	  \subfamily=subfamily}
	{
	  \DTLiffirstrow {\\\hline}{\\} \texttt{\mnemonic} & \texttt{\args} & \description
	}
      \DTLforeach*[\DTLiseq{\family}{Logic} \and \DTLiseq{\subfamily}{4}]
	{isa}
	{
	  \mnemonic=mnemonic,
	  \args=args,
	  \description=shortdescription,
	  \family=family,
	  \subfamily=subfamily}
	{
	  \DTLiffirstrow {\\\hline}{\\} \texttt{\mnemonic} & \texttt{\args} & \description
	}\\\hline
    \end{tabu}
  \caption{Logic instructions}
  \label{tbl:logic_instructions}
  \end{center}
\end{table}

\begin{table}
  \begin{center}
    \begin{tabu} to \textwidth {|ll|X[l]|}
      \hline
      \multicolumn{2}{|c|}{Mnemonic} & Description
      \DTLforeach*[\DTLiseq{\family}{Control} \and \DTLiseq{\subfamily}{0}]
	{isa}
	{
	  \mnemonic=mnemonic,
	  \args=args,
	  \description=shortdescription,
	  \family=family,
	  \subfamily=subfamily}
	{
	  \DTLiffirstrow{\\\hline\hline}{\\} \texttt{\mnemonic} & \texttt{\args} & \description
	} 
      \DTLforeach*[\DTLiseq{\family}{Control} \and \DTLiseq{\subfamily}{1}]
	{isa}
	{
	  \mnemonic=mnemonic,
	  \args=args,
	  \description=shortdescription,
	  \family=family,
	  \subfamily=subfamily}
	{
	  \DTLiffirstrow {\\\hline}{\\} \texttt{\mnemonic} & \texttt{\args} & \description
	}
      \DTLforeach*[\DTLiseq{\family}{Control} \and \DTLiseq{\subfamily}{2}]
	{isa}
	{
	  \mnemonic=mnemonic,
	  \args=args,
	  \description=shortdescription,
	  \family=family,
	  \subfamily=subfamily}
	{
	  \DTLiffirstrow {\\\hline}{\\} \texttt{\mnemonic} & \texttt{\args} & \description
	}
      \DTLforeach*[\DTLiseq{\family}{Control} \and \DTLiseq{\subfamily}{3}]
	{isa}
	{
	  \mnemonic=mnemonic,
	  \args=args,
	  \description=shortdescription,
	  \family=family,
	  \subfamily=subfamily}
	{
	  \DTLiffirstrow {\\\hline}{\\} \texttt{\mnemonic} & \texttt{\args} & \description
	}\\\hline
    \end{tabu}
  \caption{Control instructions}
  \label{tbl:control_instructions}
  \end{center}
\end{table}

\begin{table}
  \begin{center}
    \begin{tabu} to \textwidth {|ll|X[l]|}
      \hline
      \multicolumn{2}{|c|}{Mnemonic} & Description
      \DTLforeach*[\DTLiseq{\family}{Data type conversion}]
	{isa}
	{
	  \mnemonic=mnemonic,
	  \args=args,
	  \description=shortdescription,
	  \family=family,
	  \subfamily=subfamily}
	{
	  \DTLiffirstrow{\\\hline\hline}{\\} \texttt{\mnemonic} & \texttt{\args} & \description
	}\\\hline
    \end{tabu}
  \caption{Data type conversion instructions}
  \label{tbl:data_type_conversion_instructions}
  \end{center}
\end{table}


\subsubsection{Conventions for semantic descriptions}
\label{sssec:semantic_conventions}

The following sections gives the detailed specification of the instructions set. For the semantic descrition
a high level formal language is adopted. It is worth to mention that as any high level language,
there are more than one way to express the same operation. Since that, the instruction descriptions
are arbitrary.

\subsubsection{Predefined variables}
\label{sssec:predefined_variables}
They are used to refer parts of the procesor.

\begin{itemize}
\item $pc$: Program counter, stores 32 bits unsigned integers. It is incremented in 4 positions since
      it indexes instruction words of 32 bits width. It means that two consecutive instructions in the program memory are in the addresses
      pc and pc + 4. The pc content must be always a multiple of 4.
\item $m[]$: Array that represents the data memory. It is addressed by bytes (8 bits) but it can
      returns bytes (8 bits), halfs (16 bits) and words (32 bits). As mentioned in section \ref{ssec:data_memory}, halfs and words must be
      aligned.
\item $im[]$: Array that represents the instruction memory. It is addressed by words (32 bits).
\item $alu_{x}.c$, $alu_{x}.v$, $alu_{x}.z$, $alu_{x}.n$: Repersent ALU operation output flags ($x=0$ is for low GPR and $x=1$ is for high GPR).
\item $r_{x}$, $r_{y}$, $r_{z}$: Represent any of the 32 general purpose integer registers R0, $\dots$, R31.
\item $r_{x}.c$, $r_{x}.v$, $r_{x}.z$, $r_{x}.n$: Represent the associated flags to register $r_{x}$.
\item $fpu_{x}.a$, $fpu_{x}.i$, $fpu_{x}.z$, $fpu_{x}.n$: Repersent FPU operation output flags ($x=0$ is for low FPR and $x=1$ is for high FPR).
\item $f_{x}$, $f_{y}$, $f_{z}$: Represent any of the 32 floating point registers F0, $\dots$, F31.
\item $f_{x}.a$, $f_{x}.i$, $f_{x}.z$, $f_{x}.n$: Represent the associated flags to register $f_{x}$.
\item $imm_{n}$, $disp_{n}$, $nib_{n}$: Represent constants included in instructions. ``$n$'' indicates the bit width of
      the constants.%, 16 bits for I-type instructions, 21 bits for B-type instructions, 26 bits for J-type instructions and 3 bits for C-type instructions.
\item $cnv_{x}.c$, $cnv_{x}.v$, $cnv_{x}.z$, $cnv_{x}.n$, $cnv_{x}.a$, $cnv_{x}.i$: Represent CU operation output flags ($x=0$ is for low GPR or FPR
      and $x=1$ is for high GPR or FPR).
\item $cordic_{x}.c$, $cordic_{x}.v$, $cordic_{x}.z$, $cordic_{x}.n$, $cordic_{x}.a$, $cordic_{x}.i$: Represent Cordic operation output flags
      ($x=0$ is for low GPR or FPR and $x=1$ is for high GPR or FPR).
\end{itemize}

\subsubsection{Numbers}
\label{sssec:isa_numbers}
When the description of the instruction implies the use of a number, it can be expressed in the proper radix
in order to clarify the notation. The convention adopted is like in C programming language. The native type
is radix 10. Radix 16 is noted by the prefix 0x and binary numbers by 0b.

\subsubsection{Literals}
\label{sssec:isa_literals}
Logic values are indicated by the symbols `0'\ and `1'.

\subsubsection{Operators and expressions}
\label{sssec:isa_operators_expressions}
\begin{itemize}
\item Assignments: The used symbol is ``$\longleftarrow_n$'', where $n$ represents the number of bits of the assigment.
\item Arithmetic and logic operators: They can be used for representing any operation between predefined variables and GPR's / FPR's. Addition (``$+$''),
      subtraction (``$-$''), multiplication (``$*$''), division (``$/$''), remainder (``$\%$''), logic AND (``\&''), logic OR (``\textbar''),
      logic XOR (``$\wedge$''), logic inversion (``$\sim$''), sine (``$sin$''), cosine (``$cos$''), hyperbolic sine (``$sinh$''),
      hyperbolic cosine (``$cosh$''), arctangent (``$atan$''), natural logarithm (``$ln$''), square root (``$sqrt$''), A? (``$a?$''), B? (``$b?$'').
\item Comparison: They return ``true'' or ``false'' to be used in flow control statements like
      ``if''. Equal than (``$==$''), not equal than(``$!=$''), less than (``$<$''), less or equal than (``$<=$''),
      greater than (``$>$''), greater or equal than (``$>=$'').
\item Logic concatenation: It permits to concatenate any expression of the same type. It is noted by ``\#''.
\item Range of bits: $V[$x:y$]$ represents a range of bits of the variable $V$ from the position $x$ to the
      position $y$.
\item Particular bit: $V[$x$]$ represents a particular bit of the variable $V$ in the position $x$.
\item Logic extension: $x_{n}(V)$ extends the variable $V$ with left side zeros to fit a $n$ bits size.
\item Signed extension: $sx_{n}(V)$ extends the variable $V$ with its most significant bit to fit a $n$ bits size.
\item Unsigned casting: $U_n(V)$, the variable $V$ is as an $n$ bits unsigned integer.
\item Logic replication: $n(V)$, the variable $V$ is replicated by $n$ at bit level.
\item Decision statement: the only one defined statement is ``if ( $<$logic\_condition$>$) then $<$expressions$>$ [else $<e$xpressions$>$ endif]''.
\end{itemize}

\subsection{Source and destination operands}
\label{ssec:isa_operands}

As a general overview, the operands have the functions described in table \ref{tbl:operands_functions}.

\begin{table}[h]
\begin{center}
\begin{tabular}{|c|c|c|l|}
\hline
\textbf{Type} & \textbf{Subtype} & \textbf{Group} & \textbf{Operation} \\
\hline
\hline
0 & 0 & 0 & $pc \longleftarrow pc + disp$                  \\
  &   & 1 & No arguments                                  \\
  &   & 2 & No arguments                                   \\
\hline
\hline
1 & 0 & 0 & $pc \longleftarrow r_x$                        \\
  &   & 1 & $pc \longleftarrow pc + displ$                 \\
\hline
  & 1 & 0 & $r_x \longleftarrow imm$                       \\
  &   & 1 & $r_x \longleftarrow pc + 1$                    \\
\hline
\hline
2 & 0 & 0 & $r_{x} \longleftarrow r_{y}$                       \\
  &   & 1 & $r_{x} \longleftarrow r_{y}$                       \\
  &   & 2 & $r_{x} \longleftarrow r_{y}$                       \\
  &   & 3 & $r_{x} \longleftarrow r_{y} \dagger imm$           \\
  &   & 4 & $r_{x} \longleftarrow m[r_{y} + dis]$              \\
  &   & 5 & $r_{x} \longleftarrow pc; pc \longleftarrow r_{y}$ \\
\hline
  & 1 & 0 & $mem[r_{y} + dis] \longleftarrow r_{x}$            \\
\hline
\hline
3 & 0 & 0 & $r_{x} \longleftarrow r_{y} \dagger r_{z}$         \\
\hline
\end{tabular}
\end{center}
\begin{center}
\footnotesize{\textbf{Note:} ``$\dagger$'' represents any arithmetic or logical operation.} 
\end{center}
\caption{Operands functions in instructions}
\label{tbl:operands_functions}
\end{table}


\section{Full instruction set description}
\label{sec:full_isa_description}
This section will give a full description of all the instructions opcodes and their behavior and manipulation of architecture resources.
The listing will be ordered by type, subtype and group. As a first step, all executed instructions load it's opcode to the IR. This is
noted as $ir \longleftarrow_{30} im[pc]$, and this expression should be prepended to all operations noted within the following
description.

\DTLforeach*{isa}{
  \type=type,
  \mnemonic=mnemonic,
  \args=args,
  \shortdescription=shortdescription,
  \description=description,\coding=coding,
  \operation=operation,
  \flags=flags,
  \notes=notes,
  \assemblertranslationsandvalidations=assemblertranslationsandvalidations,
  \cpuvalidations=cpuvalidations}
  {%
    \subsubsection{\mnemonic \enspace - \shortdescription}
      \label{sssec:\mnemonic}
      \begin{tabu} to \textwidth {lcX[l]}
	Mnemonic             & : & \texttt{\mnemonic} \texttt{\args} \\
	Coding               & : & \texttt{\coding} \\
	Operation            & : & \begin{tabu} to \textwidth {X[l]} \operation \end{tabu} \\
	Assember Validations & : & \texttt{\assemblertranslationsandvalidations} \\
	CPU Validations      & : & \texttt{\cpuvalidations} \\
	Affected flags       & : & \begin{tabu} to \textwidth {X[l]} \flags \end{tabu} \\
	Notes                & : & \notes \\
      \end{tabu}
  }\\