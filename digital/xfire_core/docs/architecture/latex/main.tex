\documentclass[10pt,a4paper]{article}
\setlength{\parindent}{0pt}
\setlength{\parskip}{1ex plus 0.5ex minus 0.2ex}
\addtolength{\headsep}{0.5cm}
\usepackage[a4paper]{geometry} %,left=2.5cm,right=3.5cm
\usepackage{listings}
\lstloadlanguages{VHDL}
\usepackage[colorlinks=false]{hyperref}
\usepackage[spanish, activeacute]{babel}
\usepackage[latin1]{inputenc}
\usepackage{graphics}
\usepackage{graphicx}
\usepackage{rotating}
\usepackage{color}
\usepackage{longtable}
\usepackage{latexsym}
\usepackage{multicol}
\usepackage{multirow}
\usepackage{fancyhdr}
\usepackage{bytefield}

% Paquetes matematicos
\usepackage{amsfonts}
\usepackage{amsmath}
\usepackage{amssymb}
\usepackage{amsthm}
\usepackage{mathrsfs}

\pagestyle{fancy}

%\usepackage{anysize}
%\marginsize{3cm}{4cm}{2.5cm}{3.5cm}

%---- Encabezado y pie de pagina general
\fancypagestyle{general}
{
	\fancyhf{}
	\fancyfoot{}
	\fancyhead{}
	\fancyhead[LO]{\footnotesize{\leftmark}}
	\fancyhead[RE]{\footnotesize{\rightmark}}
	\fancyhead[RO]{\footnotesize{\thepage}}
	\fancyhead[LE]{\footnotesize{\thepage}}
	\renewcommand{\headrulewidth}{0pt}
	\renewcommand{\footrulewidth}{0pt}
	}

\fancypagestyle{plain}
{
	\fancyhf{}
	\fancyfoot{}
	\fancyhead{}
	\fancyfoot[RO]{\footnotesize{\thepage}}
	\fancyfoot[LE]{\footnotesize{\thepage}}
	\renewcommand{\headrulewidth}{0pt}
	\renewcommand{\footrulewidth}{0pt}
	}

%---- Encabezado y pie de pagina de la caratula
\fancypagestyle{carat}
{
	\fancyhf{}
	\fancyfoot{}
	\fancyhead{}
	\fancyhead[LO]{}
	\fancyhead[RE]{}
	\fancyhead[RO]{}
	\fancyhead[LE]{}
	\renewcommand{\headrulewidth}{0pt}
	\renewcommand{\footrulewidth}{0pt}
	}

%---- Quito los encabezados y pies de pagina en las hojas vacias
\makeatletter
  \def\cleardoublepage{\clearpage\if@twoside \ifodd\c@page\else
  \vspace*{\fill}
    \thispagestyle{empty}
    \newpage
    \if@twocolumn\hbox{}\newpage\fi\fi\fi}
\makeatother

%---- Espaciado en el indice entre el numero y el titulo
\makeatletter
\renewcommand*{\l@subsection}{\@dottedtocline{1}{3.8em}{3.5em}}
\renewcommand*{\l@figure}{\@dottedtocline{1}{1.5em}{3.3em}}
\renewcommand*{\l@table}{\@dottedtocline{1}{1.5em}{2.8em}}
\makeatother

%---- Formateo del codigo VHDL
\lstset
{
   %extendedchars=true,%
   %labelstep=0,
   %labelsep=3ex,
   %frame=trbl,
   %frameround=tttt,
   %framespread=-3ex,
   %framerulecolor=colorFondoListado,
   %backgroundcolor=colorFondoListado,
   captionpos=b,
   breaklines=true,
   %linewidth=0.98\linewidth,
   %indent=5ex,
   tabsize=4,
   %labelstyle=\small\ttfamily,
   basicstyle=\scriptsize\sffamily,
   %numberstyle=\small\sffamily,
   identifierstyle=\scriptsize\sffamily,
   commentstyle=\scriptsize\itshape,
   stringstyle=\scriptsize\sffamily,
   keywordstyle=\scriptsize\bfseries\sffamily,
   ndkeywordstyle=\scriptsize\bfseries\sffamily,
   showstringspaces=false,
   flexiblecolumns=false
   visiblespaces=false
   %stringspaces=false
}

%---- Encabezado y pie de pagina por defecto
\pagestyle{general}

%---- Matematica
\newtheorem{defi}{Definici\'on}
\newtheorem{teor}{Teorema}
\newtheorem{prop}{Proposici\'on}
\newtheorem{lema}{Lema}
\newtheorem{cor}{Corolario}
\DeclareMathOperator{\sen}{sen}
\DeclareMathOperator{\supr}{sup}
\DeclareMathOperator{\sgn}{sgn}
\DeclareMathOperator{\lip}{lip}
\DeclareMathOperator{\codim}{codim}
\DeclareMathOperator{\gen}{gen}
\DeclareMathOperator{\im}{Im}
\DeclareMathOperator{\nuc}{Nu}
\DeclareMathOperator{\ind}{ind}
\DeclareMathOperator{\dom}{Dom}
\DeclareMathOperator{\real}{Real}
\DeclareMathOperator{\imag}{Imag}

%---- Para no separar en SILABAS se activa el comando siguiente
\hyphenpenalty=10000


\begin{document}

\renewcommand{\contentsname}{\'Indice}
\renewcommand{\partname}{Parte}
\renewcommand{\chaptername}{Cap\'itulo}
\renewcommand{\appendixname}{Ap\'endice}
\renewcommand{\bibname}{Referencias}
\renewcommand{\figurename}{Figura}
\renewcommand{\listfigurename}{\'Indice de Figuras}
\renewcommand{\tablename}{Tabla}
\renewcommand{\listtablename}{\'Indice de Tablas}

%---- Secciones preliminares
% \frontmatter
% \enlargethispage{2cm}
\thispagestyle{carat}

\

\

\

\

\begin{center}

\Large{Diseño, validación e implementación de una arquitectura RISC}

\vspace{0.4cm}

\normalsize{por}

\vspace{0.4cm}

\Large{Luciano César Natañe}

\vspace{0.4cm}

\normalsize{Tesis presentada para optar al Título de}

\vspace{0.4cm}

\large{Ingeniero Electrónico}

\vspace{0.4cm}

\normalsize{por la}

\vspace{0.4cm}

\large{Facultad de Ingeniería de la Universidad de Buenos Aires}

\

\end{center}

Director:\\
\hspace*{5.5cm} \rule{7cm}{0.5pt} \\
\hspace*{7cm} Ing. Nicolás Alvarez \\
\
Co-Director:\\
\hspace*{5.5cm} \rule{7cm}{0.5pt} \\
\hspace*{7cm} Ing. Octavio Alpago \\
\
Miembros del Jurado:\\
\hspace*{5.5cm} \rule{7cm}{0.5pt} \\
\hspace*{7cm} Ing. XXXXXXXXXXXXXXXXXXXXX \\

\hspace*{5.5cm} \rule{7cm}{0.5pt} \\
\hspace*{7cm} Ing. XXXXXXXXXXXXXXXXXXXXX \\

\hspace*{5.5cm} \rule{7cm}{0.5pt} \\
\hspace*{7cm} Ing. XXXXXXXXXXXXXXXXXXXXX \\

\

\begin{center}
Calificación: \rule{4cm}{0.5pt} \hspace{1cm} Fecha: \rule{4cm}{0.5pt}\\
\end{center}

% \newpage
% \thispagestyle{empty}
%
% \
%
% \newpage
\thispagestyle{empty}
\begin{center} 
		
		Universidad de Buenos Aires, Facultad de Ingeniería

\

Diseño, valdiación e implementación de una arquitectura RISC

\

por

\

Luciano César Natale

\

\textbf{Resúmen}

\

\end{center}

El presente trabajo constituye la Tesis de Grado necesaria para obtener el título de Ingeniero Electrónico de la Facultad de Ingeniería de la Universidad de Buenos Aires.

El objetivo de este trabajo es el diseño, la validación y la implementación de una arquitectura RISC con el objetivo de generar un núcleo de procesamiento, sintetizable en FPGA, altamente configurable y suficientemente flexible y sencillo para ser utilizado en distintas aplicaciones dentro del ámbito de la investigación  en los Laboratorios de Microelectrónica y de Sistemas Embebidos. Se presenta la teoría e historia necesaria para poder explicar las decisiones de diseño adoptadas en el desarrollo de la arquitectura. Se presentan vectores de prueba para las validaciones posteriores. Se verifica el diseño generado mediante emuladores. Se implementa el diseño en un lenguaje de descripción de hardware. Se sintetiza en FPGA el diseño y se contrastan los resultados con las emulaciones realizadas. Finalmente se analizan los resultados obtenidos, se contrastan los mismos contra trabajos similares, y se proponen trabajos futuros.
% \newpage
% \tableofcontents
% \newpage
% \thispagestyle{empty}
%
% \
%
% %---- Cuerpo principal
% \mainmatter
\section{Pipeline Implementation}

In this section the implicances and decisions of a pipelined design will be presented.
(DEFINIR FLAG DE EXCEPCION!!! Donde??? en registro especial???). 
(DEFINIR QUE SE HACE CON EL PIPE!!! Se flushea??? Se sigue ejecutando??? que consecuencias trae una u otra??? segun el Patterson, las
excepeciones son tratadas como un data hazard mas flusheando el pipe).

\subsection{Pipeline structure}
The execution of the instruction is divided in a five stage pipeline. Stages are:

\begin{enumerate}
 \item IF: Instruction Fetch
 \item ID: Instruction Decode
 \item EX: Execution
 \item MEM: Memory Access
 \item WB: Write Back
\end{enumerate}

%---- Adjuntos y anexos
\newpage
\thispagestyle{empty}

\end{document}
