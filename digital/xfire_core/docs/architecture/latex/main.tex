\documentclass[10pt,a4paper]{article}
\setlength{\parindent}{0pt}
\setlength{\parskip}{1ex plus 0.5ex minus 0.2ex}
\addtolength{\headsep}{0.5cm}
\usepackage[a4paper]{geometry} %,left=2.5cm,right=3.5cm
\usepackage{listings}
\lstloadlanguages{VHDL}
\usepackage[colorlinks=false]{hyperref}
\usepackage[spanish, activeacute]{babel}
\usepackage[latin1]{inputenc}
\usepackage{graphics}
\usepackage{graphicx}
\usepackage{rotating}
\usepackage{color}
\usepackage{longtable}
\usepackage{latexsym}
\usepackage{multicol}
\usepackage{multirow}
\usepackage{fancyhdr}
\usepackage{bytefield}

% Paquetes matematicos
\usepackage{amsfonts}
\usepackage{amsmath}
\usepackage{amssymb}
\usepackage{amsthm}
\usepackage{mathrsfs}

\pagestyle{fancy}

%\usepackage{anysize}
%\marginsize{3cm}{4cm}{2.5cm}{3.5cm}

%---- Encabezado y pie de pagina general
\fancypagestyle{general}
{
	\fancyhf{}
	\fancyfoot{}
	\fancyhead{}
	\fancyhead[LO]{\footnotesize{\leftmark}}
	\fancyhead[RE]{\footnotesize{\rightmark}}
	\fancyhead[RO]{\footnotesize{\thepage}}
	\fancyhead[LE]{\footnotesize{\thepage}}
	\renewcommand{\headrulewidth}{0pt}
	\renewcommand{\footrulewidth}{0pt}
	}

\fancypagestyle{plain}
{
	\fancyhf{}
	\fancyfoot{}
	\fancyhead{}
	\fancyfoot[RO]{\footnotesize{\thepage}}
	\fancyfoot[LE]{\footnotesize{\thepage}}
	\renewcommand{\headrulewidth}{0pt}
	\renewcommand{\footrulewidth}{0pt}
	}

%---- Encabezado y pie de pagina de la caratula
\fancypagestyle{carat}
{
	\fancyhf{}
	\fancyfoot{}
	\fancyhead{}
	\fancyhead[LO]{}
	\fancyhead[RE]{}
	\fancyhead[RO]{}
	\fancyhead[LE]{}
	\renewcommand{\headrulewidth}{0pt}
	\renewcommand{\footrulewidth}{0pt}
	}

%---- Quito los encabezados y pies de pagina en las hojas vacias
\makeatletter
  \def\cleardoublepage{\clearpage\if@twoside \ifodd\c@page\else
  \vspace*{\fill}
    \thispagestyle{empty}
    \newpage
    \if@twocolumn\hbox{}\newpage\fi\fi\fi}
\makeatother

%---- Espaciado en el indice entre el numero y el titulo
\makeatletter
\renewcommand*{\l@subsection}{\@dottedtocline{1}{3.8em}{3.5em}}
\renewcommand*{\l@figure}{\@dottedtocline{1}{1.5em}{3.3em}}
\renewcommand*{\l@table}{\@dottedtocline{1}{1.5em}{2.8em}}
\makeatother

%---- Formateo del codigo VHDL
\lstset
{
   %extendedchars=true,%
   %labelstep=0,
   %labelsep=3ex,
   %frame=trbl,
   %frameround=tttt,
   %framespread=-3ex,
   %framerulecolor=colorFondoListado,
   %backgroundcolor=colorFondoListado,
   captionpos=b,
   breaklines=true,
   %linewidth=0.98\linewidth,
   %indent=5ex,
   tabsize=4,
   %labelstyle=\small\ttfamily,
   basicstyle=\scriptsize\sffamily,
   %numberstyle=\small\sffamily,
   identifierstyle=\scriptsize\sffamily,
   commentstyle=\scriptsize\itshape,
   stringstyle=\scriptsize\sffamily,
   keywordstyle=\scriptsize\bfseries\sffamily,
   ndkeywordstyle=\scriptsize\bfseries\sffamily,
   showstringspaces=false,
   flexiblecolumns=false
   visiblespaces=false
   %stringspaces=false
}

%---- Encabezado y pie de pagina por defecto
\pagestyle{general}

%---- Matematica
\newtheorem{defi}{Definici\'on}
\newtheorem{teor}{Teorema}
\newtheorem{prop}{Proposici\'on}
\newtheorem{lema}{Lema}
\newtheorem{cor}{Corolario}
\DeclareMathOperator{\sen}{sen}
\DeclareMathOperator{\supr}{sup}
\DeclareMathOperator{\sgn}{sgn}
\DeclareMathOperator{\lip}{lip}
\DeclareMathOperator{\codim}{codim}
\DeclareMathOperator{\gen}{gen}
\DeclareMathOperator{\im}{Im}
\DeclareMathOperator{\nuc}{Nu}
\DeclareMathOperator{\ind}{ind}
\DeclareMathOperator{\dom}{Dom}
\DeclareMathOperator{\real}{Real}
\DeclareMathOperator{\imag}{Imag}

%---- Para no separar en SILABAS se activa el comando siguiente
\hyphenpenalty=10000


\begin{document}

\renewcommand{\contentsname}{\'Indice}
\renewcommand{\partname}{Parte}
\renewcommand{\chaptername}{Cap\'itulo}
\renewcommand{\appendixname}{Ap\'endice}
\renewcommand{\bibname}{Referencias}
\renewcommand{\figurename}{Figura}
\renewcommand{\listfigurename}{\'Indice de Figuras}
\renewcommand{\tablename}{Tabla}
\renewcommand{\listtablename}{\'Indice de Tablas}

%---- Secciones preliminares
% \frontmatter
% \enlargethispage{2cm}
\thispagestyle{carat}

\

\

\

\

\begin{center}

\Large{Diseño, validación e implementación de una arquitectura RISC}

\vspace{0.4cm}

\normalsize{por}

\vspace{0.4cm}

\Large{Luciano César Natale}

\vspace{0.4cm}

\normalsize{Tesis presentada para optar al Título de}

\vspace{0.4cm}

\large{Ingeniero Electrónico}

\vspace{0.4cm}

\normalsize{por la}

\vspace{0.4cm}

\large{Facultad de Ingeniería de la Universidad de Buenos Aires}

\

\end{center}

Director:\\
\hspace*{5.5cm} \rule{7cm}{0.5pt} \\
\hspace*{7cm} Ing. Nicolás Alvarez \\
\
Co-Director:\\
\hspace*{5.5cm} \rule{7cm}{0.5pt} \\
\hspace*{7cm} Ing. Octavio Alpago \\
\
Miembros del Jurado:\\
\hspace*{5.5cm} \rule{7cm}{0.5pt} \\
\hspace*{7cm} Ing. XXXXXXXXXXXXXXXXXXXXX \\

\hspace*{5.5cm} \rule{7cm}{0.5pt} \\
\hspace*{7cm} Ing. XXXXXXXXXXXXXXXXXXXXX \\

\hspace*{5.5cm} \rule{7cm}{0.5pt} \\
\hspace*{7cm} Ing. XXXXXXXXXXXXXXXXXXXXX \\

\

\begin{center}
Calificación: \rule{4cm}{0.5pt} \hspace{1cm} Fecha: \rule{4cm}{0.5pt}\\
\end{center}

% \newpage
% \thispagestyle{empty}
%
% \
%
% \newpage
\thispagestyle{empty}
\begin{center} 
		
		Universidad de Buenos Aires, Facultad de Ingeniería

\

Diseño e Implementación de un procesador de Descomposición QR aplicado a filtros de Beamforming

\

por

\

Federico Damián Camarda

\

\textbf{Resumen}

\

\end{center}

El presente trabajo constituye la Tesis de Grado necesaria para obtener el título de Ingeniero Electrónico de la Facultad de Ingeniería de la Universidad de Buenos Aires.

El objetivo de este trabajo es el diseño y la implementación en hardware digital de un procesador de descomposición QR para ser utilizado en el diseño de un filtro para el procesamiento de un \textit{Beamformer} Adaptativo. Se presenta la teoría de base necesaria para poder explicar los desarrollos realizados. Por último, se analizan los resultados obtenidos, se contrastan los mismos contra trabajos similares, y se proponen trabajos futuros.

% \newpage
% \tableofcontents
% \newpage
% \thispagestyle{empty}
%
% \
%
% %---- Cuerpo principal
% \mainmatter
% xxxxxxxxxx Estan faltando instrucciones para mover flag-to-register and register-to-flags o sino para
% xxxxxxxxxx store-flags and load-flags. Sino no hay forma de salvar el contexto.
% xxxxxxxxxx special registers, cuales pongo? ---> mascara de IEEE exceptions podria ser uno...


% ---------------------------------------------------
% Floating point flags defined by IEEE 754 standard
% ---------------------------------------------------
% invalid : when a qNAN is returned
% division by zero
% overflow: when returns +/- infinity
% underflow: when returns a denormalized
% inexact: when rounded is performed
%
% Yo use otros flags, algunos los puedo dejar si quiero, pero los IEEE deben estar para respetar el standard
%
% Agregar Floating poin exceptions
% ---------------------------------------------------






%---- Adjuntos y anexos
\newpage
\thispagestyle{empty}

\end{document}
