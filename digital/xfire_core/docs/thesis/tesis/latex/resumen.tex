\newpage
\thispagestyle{empty}
\begin{center} 
		
		Universidad de Buenos Aires, Facultad de Ingeniería

\

Diseño, valdiación e implementación de una arquitectura RISC

\

por

\

Luciano César Natale

\

\textbf{Resúmen}

\

\end{center}

El presente trabajo constituye la Tesis de Grado necesaria para obtener el título de Ingeniero Electrónico de la Facultad de Ingeniería de la Universidad de Buenos Aires.

El objetivo de este trabajo es el diseño, la validación y la implementación de una arquitectura RISC con el objetivo de generar un núcleo de procesamiento, sintetizable en FPGA, altamente configurable y suficientemente flexible y sencillo para ser utilizado en distintas aplicaciones dentro del ámbito de la investigación  en los Laboratorios de Microelectrónica y de Sistemas Embebidos. Se presenta la teoría e historia necesaria para poder explicar las decisiones de diseño adoptadas en el desarrollo de la arquitectura. Se presentan vectores de prueba para las validaciones posteriores. Se verifica el diseño generado mediante emuladores. Se implementa el diseño en un lenguaje de descripción de hardware. Se sintetiza en FPGA el diseño y se contrastan los resultados con las emulaciones realizadas. Finalmente se analizan los resultados obtenidos, se contrastan los mismos contra trabajos similares, y se proponen trabajos futuros.